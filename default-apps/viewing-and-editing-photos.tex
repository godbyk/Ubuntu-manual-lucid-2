%written by Luke Jennings ubuntujenkins@googlemail.com

%TODO not sure on the order yet, basically done but not 100% happy with it.

\section{Viewing and editing photos}

To view and edit photos in Ubuntu, you can use the \application{F-Spot} Photo
Manager application. To start F-Spot, open the \menu{Applications} menu, then
choose \menu{Graphics}, then \menu{F-Spot Photo Manager}.  When you start F-Spot
for the first time, you will see the \window{Import} window\dash how to use
this is covered in \textbf{`Importing'}.

By default, F-Spot displays your photos by date. You can view photos from a 
specific month by clicking on that month in the timeline near the top of the 
window.

You can also play slide shows of your pictures by clicking on the Play button
on the toolbar (this button looks like a green triangle).

%\screenshotTODO{Main F-Spot window}
\screenshot{03-f-spot.png}{ss:f-spot}{F-Spot lets you store, tag, and edit your photos.}

This guide often refers to the side bar on the left. If you can't see it,  
open the \menu{View} menu, then choose \menu{Components}, and choose 
\menu{Sidebar}\dash making sure the option is selected. 

\subsection{Version system}

When you edit a photo, F-Spot creates a new version so that the original is 
not lost. You can create a new version by opening the \menu{Photo} menu, then 
choosing \menu{Create New Version\ldots}. This opens the \window{Create New Version} 
window. In the \textfield{Name} text field you can type what you would like to 
call the version and then click \button{OK}. A new version will then be created.

You can view previous versions of photos by clicking on the photo that you wish 
to view, then clicking on the \button{Edit Image} button. This changes the side 
bar on the left to the ``Edit'' side bar. In the bottom left, the 
\dropdown{Version} drop-down list allows you to choose previous versions of the 
photo. 

You might want to rename a version so that you remember which version is which.
 To rename a version, click on the photo that you want to change, then click on 
the \button{Edit Image} button. This changes the side bar on the left to the 
``Edit'' side bar. In the bottom left the \dropdown{Version} drop-down list lets 
you choose the version of the photo that you want to rename. 

Open the \menu{Photo} menu, then choose \menu{Rename Version}. This will open the 
\window{Rename Version} window. Enter the new name in the \textfield{New name} 
text field, then if you want to rename the version click \button{OK}. If you don't 
want to rename the version, click \button{Cancel}.

When editing photos, you may make a mistake and may decide to remove that version
as you no longer need it. To delete a version, click on the photo that you want 
to change, then click on the \button{Edit Image} button. This changes the side 
bar on the left to the ``Edit'' side bar. In the bottom left the 
\dropdown{Version} drop-down list choose the version of the photo that you want 
to delete. Then open the \menu{Photo} menu, then choose \menu{Delete Version}. 
This will open the \window{Really Delete?} window. If you want to delete the 
version click \button{Delete}. If you don't want to delete the version, click 
\button{Cancel}.

\subsection{Importing}

When you launch F-Spot for the first time you will see the \window{Import} window.
After the first launch you can import more photos by clicking on the 
\button{Import} button. 

When you import some photos, only the photos that you have just imported are 
shown. To show all of your photos, click on the gray \textbf{X} to the right 
of the blue \textbf{Find}.

\subsubsection{Choosing where F-Spot saves photos}

When importing pictures in the \window{Import} window, the 
\checkbox{Copy files to the Photos folder} option determines where the photos 
are saved. 

If the \checkbox{Copy files to the Photos folder} option is selected then F-Spot 
will copy the photos into the \textbf{Photos} folder, which is within your 
\textbf{Pictures} folder. The pictures are then sorted by year, month and then 
date.

If the \checkbox{Copy files to the Photos folder} option is unselected then 
F-Spot will not copy the pictures into the \textbf{Photos} folder.

\subsubsection{Importing from file}

%\screenshotTODO{Import window F-Spot}

\screenshot{03-fspot-import.png}{ss:f-spot-import}{You can import all of your photos.}

To import photos that are saved on your computer, choose \dropdown{Select Folder} 
from the \dropdown{Import Source} drop-down list. This opens the \window{Import} 
window. Navigate to the folder containing your photos and click \button{Open}.

When the loading bar says ``Done Loading'' all the photos in that folder and any 
sub-folders are then displayed in the \window{Import} window. You can exclude 
importing photos from sub-folders by deselecting the 
\checkbox{Include subfolders} option.

All of the photos are imported by default, but you can choose to import only
some photos. To do so, press-and-hold the \textbf{Ctrl} key while clicking 
the photos you do not want to import. Duplicates are automatically detected when
 the \checkbox{Detect duplicates} option is selected.

You can attach tags by typing the names of the your current tags in the 
\textfield{Attach Tags} text field. If you want to use multiple tags then 
separate them with a comma.

Once you have chosen the photos that you want to import, click on the 
\button{Import} button.

\subsubsection{From digital camera}

To import photos from a digital camera, plug your camera into the \acronym{USB} port of 
your computer, and turn your camera on. If your camera is detected, Ubuntu should 
open a new window prompting you to import photos. Ensure that 
\dropdown{Open F-Spot} is chosen in the drop-down list and click \button{OK}. 
This will show the \window{Import} window. In the \dropdown{Import Source} 
drop-down list choose the option that looks like \dropdown{\ldots Camera}.
  
A \window{Select Photos to Copy from Camera\ldots} window will open. You can then 
click the photos that you want to copy. All of the photos are selected by default 
but you can add or remove individual photos by pressing-and-holding 
the \textbf{Ctrl} key while clicking on photos to deselect them. 

You can attach tags to all of them by clicking on the \checkbox{Attach tag} option
and choosing the tag in the \dropdown{Attach tag:} drop-down list. For more 
information about tags see \seclink{sec:organizing-photos}.

You can change where the files are saved in the \dropdown{Target location} 
drop-down list. The default is the \dropdown{Photos} folder \dash this is where 
F-Spot saves the photos.

Once you have chosen the photos that you want to import, click on the 
\button{Copy} button. The \window{Transferring Pictures} window should open, 
and will show the copying progress. When copying is complete, the progress bar 
will display \textbf{Download Complete}. Finally, click \button{OK} to show 
your photos in F-Spot.

\subsection{Organizing photos}
\label{sec:organizing-photos}

F-Spot makes finding photos of the same type easier by using tags. You can apply 
as many tags to a photo as you like. 

To apply tags to photos, first select the photos. Then right-click on the photos
and choose \menu{Attach Tag}. Click the tag you want add to your photos. You can 
attach tags when importing photos, as covered in the ``Importing'' section.

You can make new tags by opening the \menu{Tags} and choosing 
\menu{Create New Tag\ldots}. This will open up the \window{Create New Tag} window.
Enter the name of the tag in the \textfield{Name of New Tag:} text field. The 
\dropdown{Parent Tag:} drop-down list allows you to choose the ``parent'' tag 
for your new tag.

\subsection{Editing Images}

You may want to edit some of the photos you import into F-Spot. For example, 
you may want to remove something at the edge, some discoloring, fix red eyes, 
or straighten a photo. To edit a photo, click on the photo that you 
want to edit and then click on the \button{Edit Image} button. This changes the 
side bar on the left of the \window{F-Spot} window. The panel will show eight 
options: \button{Crop}, \button{Red-eye Reduction}, \button{De-saturate}, 
\button{Sepia Tone}, \button{Straighten}, \button{Soft Focus}, 
\button{Auto Color}, and \button{Adjust Colors}. Some of these options are 
explained in more detail in the next section.

\subsubsection{Cropping photos} 

You might want to crop a photo to change the framing or remove part of the edge 
of the photo. Click on the \button{Crop} on the left panel, then in the 
\dropdown{Select an area to crop} drop-down list choose the ratio that you 
would like to crop with. You might want choose the ratio that matches the ratio 
that you would like to print, so that the photo is not stretched.

You can create custom constraints if one of the defaults does not meet your 
requirements. This is done by choosing \dropdown{Custom Ratios} from the 
\dropdown{Select an area to crop} drop-down list. This opens the 
\window{Selection Constraints} window. Click \button{Add} to place a new entry on 
the left of the window.

Once you have chosen your constraint, move the cursor to one corner of the 
section of the photo that you want to keep. Click-and-hold the left mouse button 
and drag it to the opposite corner of the section that you want to keep. 
Release the the mouse button to finish your cropping selection.

To resize the cropping selection box, move the mouse until an arrow points to 
the side of the cropping selection box that you want to move. Click-and-hold the 
left mouse button, and move the mouse until the edge is in the right place.

%TODO is this the best way of describing this?
All ratios work in portrait and landscape mode. To change between the two, you 
need to click on the edge of the cropping selection box as if you were to resize 
the box. Moving the cursor between top right and bottom left switches between 
portrait and landscape modes.

\subsubsection{Red-eye Reduction} 

If you have taken a photo and the flash caused the subject to have 
red eyes, you can fix this problem in F-Spot. First, click on the 
\button{Red-eye Reduction} button. Move the cursor to the one corner of the 
subject's eye and click-and-hold the left mouse button as you drag the cursor to 
the opposite corner of the eye. Then, release the mouse button.

This box can be moved by placing the cursor into the middle of the red eye 
selection box until a hand cursor is shown. Then, click-and-hold the left mouse 
button and move the selection box into the correct place. When it is in the 
correct place you can release the left mouse button.

To resize the box, move the mouse until an arrow points to the side of the 
red eye selection box that you want to move. Click-and-hold the left mouse 
button, move the mouse until the edge is in the right place.

When the box covers all of the red in one eye, click the \button{Fix} button. 
You will need to repeat the process for each of the subject's eyes that is 
affected.

\subsubsection{Straighten}

If you have a photo where the subject is at an angle, you can straighten the
photo with F-Spot. First, click on the \button{Straighten} button. Then move the 
slider until the picture is straight again. F-Spot will auto crop the picture 
to remove any white parts that occur due to the rotation. When you are happy 
that the picture is straight, click on the \button{Straighten} button.

\subsubsection{Auto Color}

To automatically correct the coloring of a photo, click on the 
\button{Auto Color} button.

\subsection{Exporting to web services}

F-Spot allows you to export you photos to a Web Gallery, Folder or \acronym{CD} and the 
following services: SmugMug, Picasa Web, Flickr, 23hq and Zooomr.

You can export to these services by selecting a picture and then opening the
\menu{Photo} menu, then choosing \menu{Export to} and clicking the service that 
you require. This will open a window in which you can enter your account name and 
password for the service. This will allow you to upload pictures to this service.
