%written by Luke Jennings ubuntujenkins@googlemail.com

\section{Microblogging}

You can connect several microblogging services by opening the \menu{Applications}
menu, then choosing \menu{Internet} and then \menu{Gwibber Social Client}. 
Until you add accounts, the \window{Social Accounts} window will open.


After you have added accounts you will see the \window{Social broadcast messages} window.

In this window in the \dropdown{Add new} drop-down list you can choose the from
Flickr, Twitter, StatusNet, Qaiku, Facebook, FriendFeed, Digg, and Identi.ca.

%\screenshotTODO{Adding an Identi.ca account}

\screenshot{03-gwibber-identia-accounts.png}{ss:adding-an-identica account}{Gwibber lets you add many different account types.}

\subsection{MeMenu}

If you click your name in the top panel, you will see the ``MeMenu,'' in the box 
below your name you can type a message to post on the sites that you have set up
with Gwibber.

You can also change your account settings by clicking \dropdown{Broadcast Accounts\ldots},
 this opens the \window{Broadcast Accounts} window.


\subsection{Changing accounts}

To add more accounts after you have already added some. Click \menu{Edit} then 
\menu{Accounts}, the \window{Social Accounts} window will open.

\subsubsection{Adding accounts}

In the \window{Social Accounts} click \button{Add\ldots}, each account will need
you to enter your account details. The details that you require for each account
 is detailed as follows.

 \textbf{Flickr:} To set up a Flickr account all you need is the account login \acronym{ID}.

\textbf{Twitter:} Requires a user name and password.

\textbf{StatusNet:} A login \acronym{ID}, domain and password is needed.

\textbf{Qaiku:} You will need an \acronym{API} key, instructions for this are provided in the 
Gwibber window. You will also need your login \acronym{ID}.


\textbf{Facebook:} Click \button{Authorize}, then enter your email address and
password and click \button{Connect}. If you want to be able to post on Facebook
from Gwibber, click \button{Allow publishing}, otherwise click \button{Don't
allow}.

If you want Gwibber to show your news feed, you will need to click \button{Allow
access}, otherwise click \button{Don't allow}. You will also need to allow
status updates\dash click \button{Allow status updates}; if you don't want Gwibber
to be able to update your status, click \button{Don't allow}.

In order for Gwibber to interact with Facebook each time it is used, it will
need to have constant authorization. If not, you will have to authorize it each
time you use it. To allow constant authorization click \button{Allow}.

\textbf{FriendFeed:} A remote key is required for friend feed, Gwibber provides information 
on where to get one from. You will also need a login \acronym{ID}.

\textbf{Digg:} A login \acronym{ID} is all that is required for Digg.

\textbf{Identi.ca:} A login \acronym{ID} and password is required for Identi.ca.

\subsubsection{Removing accounts}

In the \window{Broadcast Accounts} window click the account that you want to remove 
and click \button{Remove}. 

\subsection{How Gwibber displays accounts}

Gwibber allows you to post to either all, one or a selection of accounts. This can 
be set at the bottom of the \window{Social broadcast message} window \dash each of the 
accounts that you can post with will have an icon. Clicking on an icon so that it is 
disabled (appears gray) means that you will not post to that account.

Once you have decided on which accounts you want to post to you can type your message in the 
text field above the icons, then click \button{Send}.

Each one of your accounts will have a set of icons to go with it. These icons are displayed
on the left hand size of the \window{Social broadcast message} window. The set of icons
that goes with an account has a background color. Selecting each one of these icons allows you to do
tasks for that specific account.
