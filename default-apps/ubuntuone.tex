%written by Matt Griffin me+ubuntu@mattgriffin.com

\begin{comment}

OUTLINE

[DONE] - Intro - What does Ubuntu One do
[DONE] - What are the Ubuntu 10.04 LTS features (files, notes, contacts, bookmarks, music sync)
[DONE] - How to setup? Launch the Preferences application for new and existing users, create an Ubuntu SSO account and subscribe to a plan. Add computer - add screenshot of Ubuntu One launcher under System > Preferences
[DONE] - Tour of Preferences application
[DONE] - Website features - http://one.ubuntu.com/
[DONE] - Where to get support - /support/, check blog, and read twitter/identi.ca

\end{comment}

\section{Ubuntu One}

It is common for many people to use multiple computers in the course of their work, school, and personal life. You might have a desktop at your office as well as a laptop for traveling or just going to a coffee shop. Ensuring that all of your files are accessible no matter what computer you're using is quite a difficult task. The same could be said for the complexity of keeping your Evolution address book, Tomboy notes, or Firefox bookmarks in sync.

\application{Ubuntu One} can help you keep your digital life in sync. All of your documents, music, bookmarks, address book contacts, and notes stay in sync across all of your computers. In addition, they're all stored in your personal cloud so you can use a web browser from any computer to access all of your stuff from the Ubuntu One website (\url{http://one.ubuntu.com/}).

Ubuntu One provides all Ubuntu users with 2~\acronym{GB} of storage for free. More storage capacity and contacts synchronization with mobile phones is available for a monthly fee. After you set up Ubuntu One you can continue to use your computer as you normally would, with Ubuntu One taking care of making your data appear on all your other computers with Ubuntu One installed.

\section{Setting up Ubuntu One}
To set up Ubuntu One, first open the \menu{System} menu, then choose \menu{Preferences}, then \menu{Ubuntu One}. If this is your first time running the \emph{Ubuntu One Preferences} application, it will add your computer to your Ubuntu One account.

% add screenshot of Ubuntu One launcher from System, Preferences
%\screenshotTODO{Configure and manage your Ubuntu One account}

Ubuntu One uses the Ubuntu Single Sign On (\acronym{SSO}) service for user accounts. If you don't already have an Ubuntu \acronym{SSO} account, the setup process will let you create one. When you're finished, you will have an Ubuntu \acronym{SSO} account, a free Ubuntu One subscription, and your computer will be setup for synchronization.

\section{Ubuntu One Preferences}
The Ubuntu One Preferences application shows how much of your storage capacity you are currently using as well as provides account management tools.

The \emph{Account} tab displays your account information like name and email address and links to more account management and technical support resources.

The \emph{Devices} tab lists all of the devices that are currently added to synchronize with your account. Devices are either computers or mobile phones. For the computer that you are currently using, you can adjust how much of your bandwidth is used by synchronization and connect or reconnect to Ubuntu One. You can also remove computers and mobile phones from your Ubuntu One account.

The \emph{Services} tab is where you manage what Ubuntu One features synchronize with your cloud storage and other computers. You can enable or disable the synchronization of files, purchased music, contacts, and bookmarks.

\section{More information}
For more information about Ubuntu One, its services, and technical support resources, visit the Ubuntu One website at \url{http://one.ubuntu.com/}. Follow the Ubuntu One blog at \url{http://one.ubuntu.com/blog} for news on the latest features.
