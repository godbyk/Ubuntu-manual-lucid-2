%Luke Jennings

\section{Using instant messaging}
%I haven't been able to find out if all of the protocols are a trademark or registered name, I will keep looking.

Instant messaging allows you to communicate with people you know in real time. 
Ubuntu includes the \application{Empathy} application that lets you use
instant messaging features to keep in touch with your contacts. 
To start Empathy, open the \menu{Applications} menu from the menubar, then 
choose \menu{Internet} and then \menu{Empathy IM Client}. 

Empathy lets you connect to many instant messaging networks. You can connect
to \acronym{AIM}, Gadu-Gadu, Google Talk, Groupwise, \acronym{ICQ}, 
Jabber, \acronym{MSN}, MySpace, QQ, 
\acronym{XMPP}, Sametime, Silc, \acronym{SIP}, Yahoo, or Zephyr.

\subsection{Running Empathy for the first time}

When you open Empathy for the first time you will need to configure it with 
the details of your instant messaging accounts. 

When Empathy starts you will see the \window{Welcome to Empathy} 
window. Choose the option corresponding to your situation.

\subsubsection{You have an account}

If you have an account that you have used previously with another instant 
messaging program then select the \checkbox{Yes, I'll enter my account 
details now} option. Then, click \button{Forward} to continue.

On the next screen, choose your account type from the drop-down list below
\dropdown{What kind of chat account do you have?}. Then, enter your account 
details in the field below. 

Depending on the account type that you choose, Empathy may request that you
enter a username, or an \acronym{ID} for your account, followed by a password.

%\screenshotTODO{Empathy enter user details for a google account example@googlemail.com}
\screenshot{03-empathy-new-account.png}{ss:empathy-new-account}{Creating a new instant messenger account in Empathy.}

If you do not remember your account information, you will need to visit the
website of the instant messaging network to retrieve that information.

If you have another account to add then select the \checkbox{Yes} option, 
and click \button{Forward} to repeat the above process. When you 
have entered all the accounts leave the \checkbox{No, that's all for now} 
option selected, and click \button{Apply} to finish the 
setup process. 

Next, Empathy should display the \window{Please enter personal details} screen.
If you choose to fill out this information, you will be able to communicate 
with people who are on your local network either at home or in an office. 

Enter your first name in the \textfield{First name} field, and your last name 
in the \textfield{Last name} field. Type in a way that you would like to be
identified on your local network in the \textfield{Nickname} field. When you 
have filled all of the information, click \button{Apply}.

If you don't want to communicate with people on your local network,  
select the \checkbox{I don't want to enable this feature for now} option
and click \button{Apply}.

\subsubsection{You would like an account}

If you don't have an account that you can use, then you can create one by 
selecting the \checkbox{No, I want a new account} option. Click 
\button{Forward} to display the next set of options. 

Choose the account type that you would like to create from the drop-down list
below \dropdown{What kind of chat account do you want to create?} 
You can create either a Jabber or a Google Talk account. 
\notecallout{If you wish to create another account type then you will need to 
visit the relevant website and create the account. Then follow the ``You have an account'' section.}

Next, enter the account name that you would like in the text field, and in 
the proceeding text field enter a password of your choice. If you would
like to set up another account then select the \checkbox{Yes} option, and 
repeat the above process. 

When you have entered all the accounts leave the \checkbox{No, that's all for now}
option selected, and click \button{Forward}.

Empathy should display the \window{Please enter personal details} window. 
Providing this information allows you to communicate with people who are on your 
local network either at home or in the workplace. 

Enter your \textfield{First name} in the text field, and enter your 
\textfield{Last name} in the next field. In the \textfield{Nickname} field enter
a nickname by which you would like to be identified. When you have filled all of 
the text fields click \button{Apply} to save your settings. 

If you don't want to talk to people on your local network 
then select the \checkbox{I don't want to enable this feature for now} option
and click \button{Apply}.

\subsubsection{You want to talk to people nearby}

If you would only like to communicate with people on your local network either at 
home or in the workplace, then you should select the \checkbox{No, I just want to 
see people online nearby for now} option. 

Click \button{Forward} to display the next set of options. Then enter
 your \textfield{First name} in the text field, and enter your \textfield{Last name}
 in the next field. In the \textfield{Nickname} field enter a nickname by which
you would like to be identified. When you have filled all of the text fields,
click \button{Forward}.

%\screenshotTODO{Enter details window for setting up talking to people nearby}

\screenshot{03-empathy-nearby.png}{ss:empathy-new-account-local}{You can talk to people nearby by entering your information.}

\subsection {Changing account settings}

If you need to add more accounts after the first launch, then open the 
\menu{Edit} menu, then choose \menu{Accounts}. Empathy will then display the 
\window{Accounts} window.

\subsubsection{Adding an account}

To add an account click on the \button{Add} button. Empathy should display 
some options on the right hand side of the window. Choose your account type 
from the \dropdown{Protocol} drop-down list. Next, enter your account name in 
the first text field. Then enter your password in the \textfield{Password} 
text field. Finally click on the \button{Log in} button to save and verify your 
settings.

\subsubsection{Editing an account}

You might need to edit an account if you change the password or get the password
wrong. Select the account you want to change on the left side of the 
\window{Accounts} window. Empathy should show the current settings for the 
account. Once you have made your changes, click \button{Save}.

\subsubsection{Removing an account}

To remove an account select the account on the left hand side of the window and 
click on the \button{Remove} button. Empathy should open the 
\window{Do you want to remove} window. Click on the \button{Remove} button to 
confirm that you want to remove the account, or click \button{Cancel} to keep
the account.

\subsection{Editing contacts}

\subsubsection{Adding a contact}

%TODO this bit needs some rewording, I am not sure if it is clear enough.
%Note: I'm confused here... Agree that it needs rewording. --ilya
%Agreed I think I've fixed it. Can someone check it? - JasonCook599
To add a contact open the \menu{Chat} menu, then choose \menu{Add contact}. 
Empathy should open the \window{New Contact} window. 

In the \dropdown{Account} drop-down list choose the account that you want to 
add the contact to. When creating a contact you must select a service that matches the service you contact is using.

For example if your contact's address ends in ``@googlemail.com'' then you will 
need to add it to an account that ends in ``@googlemail.com.'' Likewise if the 
contact's email ends in ``@hotmail.com'' then you would need to add it to an 
account ending in ``@hotmail.com.''

After choosing the account you wish to add the contact to, you will
need to enter their login \acronym{ID}, their username, their screen name or their email
address in the \textfield{Identifier} text field.

Then, in the \textfield{Alias} text field, enter the name that you would
like to see it in your contact list. Click \button{Add} to add the contact to
your list of contacts.

\subsubsection{Removing a contact}

Click on the contact that you want to remove and then open the \menu{Edit}
menu, then choose \menu{Contact}, then \menu{Remove}. This will
open the \window{Remove contact} window. 

Click on the \button{Remove} button to confirm that you want to remove a contact,
or click \button{Cancel} to keep the contact.

\subsection{Communicating with contacts}

\subsubsection{Text}

To communicate with a contact, select the contact in Empathy's main window and
double-click their name. Empathy should open a new window where you can type
messages to your contact, and see a record of previously exchanged messages.

To send a message to the contact, type your message in the text field below
the conversation history. 

When you have typed your message press the \keystroke{Enter} key to send the 
message to your contact. If you are communicating with more than one 
person then all of the conversations will be shown in tabs within the same window.

%TODO more detail needed here.
\subsubsection{Audio}

If your contact has audio capabilities then there will be an icon of a microphone
next to their name. Click on the microphone icon to open a popup menu. Choose the 
\menu{Audio call} option from the menu. Empathy should then open the 
\window{Call} window.

This window shows your picture on the right and your contact's picture on the left.
Ensure that your microphone and speakers are connected, and proceed with the 
audio conversation. You can finish the conversation by clicking on the 
\button{Hang up} button.

%TODO more detail needed here.
\subsubsection{Video}

If your contact has video chat capabilities then there will be an icon of a webcam
next to their name. Click on the icon to open a popup menu. Choose the 
\menu{Video call} option from the menu. Empathy should then open the
\window{Call} window. 

This window shows your webcam view in the top right and your contact's webcam 
will be in the middle.

If you don't have a webcam then your picture will be shown instead. You can finish 
the call by clicking on the \button{Hang up} button.  

\subsection{Sending and receiving files}

\subsubsection{Sending a file}

When you are in a conversation with a contact and you would like to send them a 
file, open the \menu{Contact} menu and then choose \menu{Send file}. 

Empathy should open the \window{Select file} window. Find the file that 
you wish to send and click on the \button{Send} button. A \window{File Transfers} 
window will open showing the chosen file and its transfer progress. 

When the file transfer is complete, you can close the 
\window{File Transfers} window.

\subsubsection{Receiving a file}

When a contact wants to send you a file, the status icon to the left of the 
contact's name will flash with an icon of a paper plane. 

To receive the file double-click the contact's name. Empathy will open the 
\window{Select a destination} window. Choose a location where you would like
Empathy to save the file, and click \button{Save}. Empathy should 
open the \window{File Transfers} window. 

The \window{File Transfers} window shows you the progress of current file 
transfers. You can stop file transfers by clicking on the \button{Stop} button, 
open transferred files by clicking on the \button{Open} button, and clear the 
list of completed transfers by clicking on the \button{Clear} button.

\subsection{Changing your status}

You can use your status to show your contacts how busy you are or what you are 
doing. You can use the standard statuses, which are ``Available,'' ``Busy,''  ``Away,'' ``Invisible,'' and ``Off-line.''  These can be changed in the main Empathy window 
from the drop-down list at the top of the window. 

The same drop-down list lets you set a custom status by choosing 
``Custom Message\ldots'' next to the icon that matches your status. 
Type what you would like your status to say, and click on the green check mark.

\subsection{Changing your picture}

Your picture is what your contacts will see next to your name in their contact list.
The default picture is the outline of a person. You can change your picture by 
opening the \menu{Edit} menu, then choosing \menu{Personal Information}. 

Empathy should open the \window{Personal Information} window. From the 
\dropdown{Account} drop-down list choose the account that you want to change, 
then click on the picture on the right hand side of the window. 

Empathy should open the \window{Select Your Avatar Image} window.
Find the file containing your picture, and click \button{Open}. 
If you would like to return it to the default avatar, click on the 
\button{No Image} button instead.
