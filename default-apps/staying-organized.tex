\section{Staying organized}
%  [intro to using evolution with or without an account]

The \application{Evolution} application in Ubuntu can let you keep and manage
a list of your contacts, maintain a calendar, and a task list.

If you have already set up Evolution with an email account, you do not need
to do any further setup to use these features. If you do not wish to use
Evolution for email, you can still use it for managing your contacts or 
maintaining a schedule, as well as keep track of tasks and memos that you can
create for yourself.

To start Evolution, open the \menu{Applications} menu, then choose 
\menu{Office} and then \menu{Evolution Mail and Calendar}.

%FUTURE: may wish to elaborate on not choosing an account if running for the 
%first time.

\subsection{Managing your contacts}

%\screenshotTODO{Evolution showing the contacts section}

\screenshot{03-contacts-evolution.png}{ss:evolution-contacts}{You can view, edit, and add contacts.}

If you would like to keep a list of your contacts \dash personal or 
professional contact information for people and organizations \dash you can 
manage these contacts in Evolution.

To view contacts, click on the \button{Contacts} button below the folder
list on the left side of the Evolution window. The folder list on the left
will be replaced by a list of address book types. Click on an address book,
for example ``Personal.''

\marginnote{An address book is a collection of contacts and contact lists. 
It can either be stored on your computer, or on a remote server.}
The right side of the window will display a list of contacts. Click a contact
to show the contact's details in the lower portion of the right side of the
window.

\marginnote{Ubuntu One is a free service you can use to sync and store contacts, as well as other information. For more information on Ubuntu One see the dedicated section later in this chapter.}
If you use \application{Ubuntu One}, you may have two address books 
\dash a ``Personal'' address book stored on your computer, 
and an ``Ubuntu One'' address book. You can add contacts to either address
book, though only the ``Ubuntu One'' address book is synchronized to your
Ubuntu One account.

 \subsubsection{Searching for contacts}

To find a contact, type in a few a few letters from the contact's first or
last name in the search text box on the upper right of the window, and press
\keystroke{Enter}. The list below should change to only show contacts whose
name matches your search term.

 \subsubsection{Adding or editing a contact}

To make changes to an existing contact, find the contact in the list and 
double-click on the entry. Evolution should open a \window{Contact Editor} 
window for the contact.

Switch between the different tabs in the contact editor to make changes to the
contact. Click \button{OK} when you have finished making your changes.

To add a new contact, click on the \button{New} on the toolbar. Evolution should
open the \window{Contact Editor} window. Enter the contact's details in the 
contact editor window, and click \button{OK} when finished.

\subsection{Managing your schedule}

If you like to manage your schedule with a computer, you can maintain this 
schedule in Ubuntu using Evolution.

To view your calendar, click on the \button{Calendars} button below the folder
list on the left side of the Evolution window. The folder list on the left
will be replaced by a list of calendars, and a mini-calendar showing the 
current month.

Evolution allows you to manage more than one calendar. For example, you could
have a personal calendar and a school or work calendar. You can also 
subscribe to the calendar of a friend or family member who may choose to share
their calendar with you.

Click on one of the calendars in the list. By default, you should have a 
``Personal'' calendar in the list. The middle of the window should now
show a view of the current day, showing all the hours of the current day.

If the calendar already has some events, Evolution will show the event
in the day view between the hours when the event starts and finishes. 
You can double-click on the event to open its details, or drag the event to
a different time or date to reschedule it.

In the day view, you can click on a different day on the mini-calendar on the
left side of the screen. Evolution will then display that day in the day view.

You may also wish to see more than one day at a time. This will allow you to 
compare schedules on different days, or find a free day for an event you 
wish to schedule. In Evolution, you can click on the \button{Work Week} or
\button{Week} buttons on the toolbar to see an entire week at the same time.
Click on the \button{Month} button on the toolbar to see a view of the entire
month \dash if an event is difficult to read due to the small space allotted
to each day, you can hover your mouse over the event to have Evolution 
show the full title of the event. Finally, the \button{List} button on the 
toolbar shows upcoming appointments in a list, allowing you to see all of 
your upcoming appointments at a glance.

On the right side of the window, Evolution displays a list of tasks and memos.
You can add a new task or memo to Evolution

 \subsubsection{Adding a new event}

The simplest way to add a new task is to click a time in the day view, and
begin typing. An event ``bubble'' will appear, containing the text that you
are typing. If you want to add a longer event, drag your mouse from the first
time slot to the last before starting to type.

%\screenshotTODO{visual example of adding an event in the day view by typing}
\screenshot{03-evolution-event.png}{ss:evolution-event}{You can stay organized by adding events to your calendar.}

To add a new event without using the day view, click on the \button{New} 
button on the toolbar. Evolution should open the \window{Appointment} window.
In the \textfield{Summary} field, enter a short title for the event as you
want it to appear on the calendar. Optionally specify the location and enter
a longer description if you would like. Make sure that the time and date, as
well as the duration, are as you want them. Finally, click on the 
\button{Save} button on the toolbar to save this new event (the button looks
like a hard drive, and is the first button on the toolbar).

 \subsubsection{Scheduling a meeting}

If you would like to schedule a meeting with one of your contacts, Evolution
can assist you in sending out an invitation and processing replies.

To create a meeting invitation, choose \menu{File \then New \then Meeting}
from the menubar. Specify the subject, location, time and duration, and 
description as when you create a regular event. 

You will then need to add attendees to this meeting. To add an attendee,
click on the \button{Add} button. In the list of attendees, Evolution will
add a new row \dash type the attendee's email address or contact name.

When you are finished adding attendees, click on the \button{Save} button on
the toolbar. Evolution should then ask you if you would like to send meeting
invitations to your selected participants. Click \button{Send} to send out
these invitations. The invitations will be sent the next time you check
email in Evolution.

If your contact chooses to reply to the meeting invitation, Evolution will
show you a new email message. In the body of the email message, Evolution will
display an \button{Update Attendee Status} button. Click on that button to
mark your contact as attending the meeting.


