\section{Working with documents, spreadsheets, and presentations}

Quite often, you may need to use your computer for work. You may have a need 
to use a word processor to write a document. You may need to work on a 
spreadsheet, do calculations on a table of data or create a data chart. You 
may want to work on slides for a presentation.

In Ubuntu, you can use the OpenOffice.org suite of applications for these 
tasks.

\subsection{Working with documents}

\marginnote{The OpenOffice.org Word Processor is also known as the 
OpenOffice.org Writer. Spreadsheet is also known as Calc, and Presentation is
known as Impress.}
If you need to work with documents, you can use the OpenOffice.org Word 
Processor. To start the word processor, open the \menu{Applications}
menu, choose \menu{Office}, and then choose \menu{OpenOffice.org Word
Processor}. Ubuntu should then open the main window for the word processor.

\subsection{Working with spreadsheets}

If you need to work with spreadsheets, you can use the OpenOffice.org 
Spreadsheet. To start the spreadsheet application, open the 
\menu{Applications} menu, choose \menu{Office}, and then choose
\menu{OpenOffice.org Spreadsheet}.

\subsection{Working with presentations}

If you need to work with slides for a presentation, you can use the 
OpenOffice.org Presentation. To start the presentation application, open the 
\menu{Applications} menu, choose \menu{Office}, and then choose
\menu{OpenOffice.org Presentation}.

\subsection{Getting more help}

Each of these applications comes with a comprehensive set of help screens.
If you are looking for more assistance with these applications, press the
\keystroke{F1} key after starting the application.

