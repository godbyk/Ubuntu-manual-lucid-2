%written by Luke Jennings ubuntujenkins@googlemail.com

\section{Taking notes}

You can take notes in a program called \textbf{Tomboy Notes}. You can use it to 
make a shopping or a to do list. Click \menu{Applications}, then click 
\menu{Accessories} and click \menu{Tomboy Notes}.

%\screenshotTODO{Screenshot of main Tomboy window}

\screenshot{03-tomboy.png}{ss:tomboy-notes}{You can record information that you need to remember.}

You can search all of your notes by typing a word in the \textfield{Search:} text 
field in the main tomboy window.

\subsection{Making notes}

To create a new note click \menu{File}, then click \menu{New}. The \window{New Note}
 window will open.

The \window{New Note} window will contain a blue title ``New Note'' \dash this can be 
deleted and changed to a title that makes the note more memorable. The main content 
of the note can be typed where it says ``Describe your new note here.'' Once you have 
entered your text just close your note as all changes are automatically saved.

To delete the note click the red delete note button. This will open a \window{Really delete this note?} 
window. If you do want to delete the note click the \button{Delete} button, otherwise
click the \button{Cancel} button.

You can add a note to a notebook by clicking the \button{Notebook} button and clicking
the option next to the notebook that you want to move the note to.


\subsection{Organizing notes}

You can organize your notes in Tomboy using ``Notebooks.'' This makes finding you 
notes quicker and in a more logical location. To create a new note book click \menu{File}, 
then \menu{Notebooks}, and click \menu{New Notebook\ldots}. 

The \window{Create a new notebook} window will open, type the name of the notebook 
in the \textfield{Notebook name:} text field. Once you have typed the notebook name
click the \button{Create} button.

The notebook will now show up in the sidebar of Tomboy Notes. You can click and hold 
on the note of your choice and drag it on top of the notebook that you want to move it to.

\subsection{Synchronizing}

You can synchronize your notes with your Ubuntu One account, which means that you 
can access them across all of your Ubuntu computers. You can also access them from 
\url{https://one.ubuntu.com/}.
%TODO Uncomment when the ubuntu one section is written.
%\marginnote{For more information on ubuntu one see \chaplink{ch:}

To synchronize your notes click the \menu{Edit}. Then click \menu{Preferences}.
This will open the \window{Tomboy Preferences} window. Click the \textbf{Synchronization} 
tab and then in the \dropdown{Service} drop down click \dropdown{Tomboy Web}. 

Next click the \button{Connect to Server} button. This will open a web page in 
\textbf{Firefox} you will need to enter the email address that you use for Ubuntu 
One and your password. Then click the \button{Continue} button, then in the \textfield{Computer Name}
text field enter a name that reminds you of that computer and click the \button{Add This Computer}
button. Firefox will then display a page that says something similar to ``Tomboy Web 
Authorization Successful.''

Back at the \window{Tomboy Preferences} window click the \button{Save} button.
A new window will pop up asking if you want to ``synchronize your notes now.''
Click the \button{Yes} button and the \window{Synchronizing Notes\ldots} window 
will show. Once the synchronization is complete click the \button{Close} button.

If you want to synchronize the notes again click \menu{Tools} and click \menu{Synchronize Notes}.
Your notes will start synchronizing. When they are done, click the \button{close}
button.
