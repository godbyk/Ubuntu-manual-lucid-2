\begin{comment}

  Most core Firefox help content here is adapted from Firefox 
  help documentation, licensed under the CC-BY-SA 3.0 license.

  %http://support.mozilla.com/en-US/kb/Browsing+basics?style_mode=inproduct

  The following people are listed for attribution at the time of copy:

    * Chris\_Ilias
    * Bo
    * underpass
    * mozilla\_help\_viewer\_project

  %http://support.mozilla.com/en-US/kb/Tabbed+browsing?style_mode=inproduct

  The following people are listed for attribution at the time of copy:

    * Bo
    * jehurd
    * cl58
    * underpass
    * kjhass
    * djstsys
    * Chris\_Ilias
    * mozilla\_help\_viewer\_project

\end{comment}

\section{Browsing the web}

Once you have connected to the Internet, you should be able to browse
the web with Ubuntu. Mozilla \application{Firefox} is the default 
application for browsing the web in Ubuntu.

\screenshot[t]{03-ubuntu-start-firefox.png}{ss:firefox-home-page}{The default Ubuntu home page for the Firefox web browser.}% Putting this here so LaTeX places it in a nice spot

\subsection{Starting Firefox}

\marginnote{To set other keyboard shortcuts or to change the shortcut for launching Firefox, go to \menu{System \then Preferences \then Keyboard Shortcuts}.}
To start Firefox, click \menu{Applications \then Internet \then Firefox Web Browser}. If your keyboard has a ``\acronym{WWW}'' button, you can also press that button to start Firefox.

\subsection{Navigating web pages}

\subsubsection{Viewing your homepage}

When you start Firefox, you will see your home page. By default, you will 
see the Ubuntu Start Page.

To go to your home page quickly, press \keystroke{Alt+Home}.

\subsubsection{Navigating to another page}

\marginnote{\acronym{URL} stands for uniform resource locator and \acronym{WWW} stands for world wide web.}
To navigate to a new web page, you need to type its Internet address 
(also known as a \acronym{URL}) into the Location Bar. \acronym{URL}s normally begin with 
``http://'' followed by one or more names that identify the address. 
One example is ``\url{http://www.ubuntu.com/}''.

\screenshot{03-firefox-location-bar.png}{ss:firefox-location-bar}{You can enter a web address or search the Internet by typing in the location bar.}\clearpage

To navigate:

\begin{enumerate}
   \item Click on the Location Bar to select the \acronym{URL} that is already there.
   \item Type the \acronym{URL} of the page you want to visit. The \acronym{URL} you type 
replaces any text already in the Location Bar.
   \item Press \keystroke{Enter}. 
\end{enumerate}

\marginnote{You can also press \keystroke{F6} on your keyboard to highlight the location bar in Firefox.}
To quickly select the \acronym{URL} of the Location Bar, press \keystroke{Ctrl+L}.

If you don't know a \acronym{URL}, try typing something specific to the page you want to 
visit (for example a name or other search request) into the Location Bar and 
press \keystroke{Enter}. This will search your preferred search engine\dash 
Google by default\dash for that term, and take you to the web page that is
the top result from the search.

\subsubsection{Clicking a link}

Most web pages contain links you can click to move to other pages.

To click a link:

\begin{enumerate}
  \item Move the mouse pointer until it changes to a pointing finger. 
This happens whenever the pointer is over a link. Most links are underlined 
text, but buttons and pictures on a web page can also be links.
  \item Click on the link once. While Firefox locates the link's page, 
status messages will appear at the bottom of the window.
\end{enumerate}

\subsubsection{Retracing your steps}

\marginnote{To go backwards and forwards you can also use \keystroke{Alt+Left}to go backwards or \keystroke{Alt+Right} to go forwards.}
If you want to visit a page you have seen before, there are several 
ways to do so.

\begin{itemize}
  \item To go back or forward one page, click on the \button{Back} or 
\button{Forward} button.
  \item To go back or forward more than one page, click on the small triangle 
next to the \button{Forward} button. You should see a list of 
pages you've recently visited. To return to a page, select it from the list.
  \item To see a list of any \acronym{URL}s you've typed into the Location Bar, 
click on the down arrow at the right end of the Location Bar. To view a page, 
select it from the list.
  \item To choose from pages you've visited during the current session, 
open the \menu{History} menu and choose from the list in the bottom section 
of the menu.
  \item To choose from pages you've visited during the past several sessions, 
open the \menu{History} menu and choose \menu{Show All History}. 
Firefox should open a \window{Library} window, which shows a list of folders. 
Click on the folders to displays sub-folders, or titles of web pages you've 
visited in the past. Click on a page's title to view that page.
\end{itemize} 

\subsubsection{Stopping and reloading}

If a page is loading too slowly or you no longer wish to view a page, 
click on the \button{Stop} button.

To reload the current page or to get the most up-to-date version, 
click on the \button{Reload} button or press \keystroke{Ctrl+R}. 

\subsubsection{Opening new windows}

At times, you may want to have more than one browsing window. This may help
you organize your browsing session better, or separate web pages that you 
are viewing for different reasons. 

There are two ways to create a new window:

\begin{itemize}
  \item On the menubar, open the \menu{File} menu, then choose 
\menu{New Window}.
  \item Press \keystroke{Ctrl+N}.
\end{itemize}

Once a new window has opened, you can use it just like the first window \dash 
including navigation and opening tabs.

\subsubsection{Opening a link in a new window}

Sometimes, you may want to click on a link to navigate to another web page,
but do not want the original page to close. To do this, you can open the 
link you'd like to click in its own window.

There are two ways to open a link in its own window:

\begin{itemize}
  \item Right-click on a link to open its popup menu. Choose the
\menu{Open Link in New Window} option. A new window will open, containing
the web page for the link you clicked.
  \item Press-and-hold the \keystroke{Shift} key while clicking a link. This will also open the web page in a new window.
\end{itemize}

\subsection{Tabbed browsing}

\marginnote{You can alternate quickly between different tabs by using the keyboard shortcut \keystroke{Ctrl+Tab}.}
If you would like to visit more than one web page at a time, you can use 
\emph{Tabbed Browsing} to navigate the web.

Tabbed browsing lets you open several web pages within a single Firefox 
window, each displaying in its own tab. This frees up space on your 
desktop since you don't have to have a window open for every web page 
you're currently visiting. You can open, close, and reload web pages 
in one place without having to switch to another window.

\subsubsection{Opening a new blank tab}

There are three ways to create a new blank tab:

\begin{itemize}
  \item Click on the \button{New Tab} button on the right side of the last tab.
  \item On the menubar, open the \menu{File} menu, and then choose 
\menu{New Tab}.
  \item Press \keystroke{Ctrl+T}.
\end{itemize}

When you create a new tab, it will contain a blank page with the Location Bar 
focused. Start typing a web address (\acronym{URL}) or other search term to open a 
website in the new tab.

\subsubsection{Opening a link in its own tab}

Sometimes, you may want to click on a link to navigate to another web page,
but do not want the original page to close. To do this, you can open the 
link you'd like to click in its own tab.

There are many ways to open a link in its own tab:

\begin{itemize}
  \item If your mouse has a middle button, or a wheel, click on the link with 
the middle mouse button or wheel. A new tab should open, containing the web
page for the link you clicked.
  \item Click on the link with the left mouse button, and keep holding down 
the mouse button. Drag the link up to a blank space on the tab bar, and release
the mouse button. A new tab should open, containing the web page for the link
you dragged.
  \item Press-and-hold the \keystroke{Ctrl} key while clicking the left mouse 
button on the link. A new tab should open, containing the web page for the link
you clicked.
  \item Right-click on a link to open its popup menu. Choose the
\menu{Open Link in New Tab} option. A new tab will open, containing
the web page for the link you clicked.
  \item Click on a link, holding both left and right mouse buttons.
\end{itemize} 

\subsubsection{Closing a tab}

Once you are done viewing a web page in a tab, you can close that tab. 

There are four ways to close a tab:

\begin{itemize}
  \item Click on the \button{Close} button on the right side of the tab 
you want to close.
  \item On the menubar, open the \menu{File} menu, and then choose 
\menu{Close Tab}.
  \item Click on the tab you want to close with the middle mouse button, or the
mouse wheel, if you have one.
  \item Press \keystroke{Ctrl+W}.
  \item Click on the tab with both mouse buttons.
\end{itemize}

\subsubsection{Restoring a closed tab}

Sometimes, you may close the wrong tab by accident, or want to bring back a tab that you've recently closed.

To bring back a tab you've closed, do one of the following:

\begin{itemize}
  \item On the menubar, open the \menu{History} menu, choose 
\menu{Recently Closed Tabs}, and then choose the name of the tab you want to 
restore.
  \item Press \keystroke{Ctrl+Shift+T} to re-open the most recently closed tab.
\end{itemize}

\subsubsection{Changing the tab order}

To move a tab to a different location on the tab bar, drag it there using 
your mouse. Click-and-hold on the tab and drag the tab to a new place on the 
tab bar. While you are dragging the tab, Firefox will display a small 
indicator to show where the tab will be moved.

\marginnote{When moving a tab to a new window it may reload the page. remember to save your work before doing this.}

\subsubsection{Moving a tab between windows}

If you have more than one Firefox window open, you can move an open tab 
to a different window. You can also split a tab off to become its own window.

To move a tab from one Firefox window to another already open window, 
click-and-hold on the tab and drag it to the tab bar on the other Firefox 
window. When you release the mouse button, the tab will be attached to the new 
window.

To move a tab from one window into its own window, click-and-hold on the tab 
and drag the tab below the tab bar. When you release the mouse button, the tab 
will become a new window.

\subsection{Searching}

You can search the web, or other collections, from within Firefox without
first visiting the home page of the search engine.

By default, Firefox will search the web using the Google search engine.

\subsubsection{Searching the web}

To search the web in Firefox, type a few words into the Firefox search Bar.

For example, if you want to find information about the \emph{Ubuntu}:

\begin{enumerate}
   \item Click on the \menu{Search Bar}.
   \item Type the phrase ``Ubuntu.'' Your typing replaces any text currently in the Search Bar.
   \item Press \keystroke{Enter} to search.
\end{enumerate}

Search results from Google for ``Ubuntu'' should appear in the Firefox 
window. 

\subsubsection{Selecting search engines}

\screenshot{03-searchbar-firefox.png}{ss:firefox-search-bar}{These are the other search engines you can use \dash by default \dash from the Firefox search bar.}

If you do not want to use Google as your search engine in the Search Bar, 
you can change the search engine that Firefox uses.

\marginnote{The Ubuntu home page's search bar uses Google by default, but will automatically use Yahoo if Yahoo is selected in the Search Bar.}

To change the search engine, click on the icon on the left side of the Search 
Bar. Choose one of the other search engines in the list. Some search engines, 
like Google, search the whole web; others, like Amazon.com, only search 
specific sites. 

%FUTURE: decide whether to cover getting add'l search engines

\subsubsection{Searching the web for words selected in a web page}

Sometimes, you may want to search for a phrase that appears on a different web
page. Instead of copying and pasting the phrase into the Search Bar, 
Firefox allows you to search the web for words you select within a web page.

\begin{enumerate}
  \item Highlight any words in a web page using your left mouse button.
  \item Right-click on the text you've highlighted to open a popup menu.
Choose the option \menu{Search [Search Engine] for 
``[your selected words]''}. 
\end{enumerate}

Firefox should open a new tab containing search results for your
highlighted words, found using the currently selected search engine.

\subsubsection{Searching within a page}

\screenshot{03-firefox-find-bar.png}{ss:firefox-find-toolbar}{You can search within web pages using the \button{Find Toolbar}.}

You may want to look for specific text within the web page you are viewing.
To find text within the current page in Firefox:

\begin{enumerate}
   \item Press \keystroke{Ctrl+F} or choose \menu{Edit \then Find} to open the 
\textfield{Find Toolbar} at the bottom of Firefox.
   \item Enter the text you want to find into the \button{Find} field in the
Find Toolbar. The search automatically begins as soon as you type something 
into the field.
   \item Once some text has been matched on the web page, you can:
     \begin{itemize}
       \item Click \button{Next} to find text in the page that is below the 
current cursor position.
       \item Click \button{Previous} to find text that is above the current 
cursor position.
       \item Click on the \button{Highlight all} button to highlight 
occurrences of your search words in the current page.
       \item Select the \checkbox{Match case} option to limit the search to 
text that has the same capitalization as your search words.
     \end{itemize}
\end{enumerate}

To find the same word or phrase again, press \keystroke{F3} or choose 
\menu{Edit \then Find Again} from the menubar. 

\subsection{Viewing web pages full screen}

To display more web content on the screen, you can use \emph{Full
Screen mode}. Full Screen mode condenses the Firefox's toolbars into one
small toolbar. To enable Full Screen mode, simply choose \menu{View \then Full
Screen} or press \keystroke{F11}.

\subsection{Copying and saving pages}

With Firefox, you can copy part of a page so that you can paste it elsewhere,
or save the page or part of a page as a file on your computer.

\subsubsection{Copying part of a page}

To copy text from a page:

\begin{enumerate}
  \item Highlight the text and/or images with your mouse.
  \item Choose \menu{Edit \then Copy} from the menubar or press \keystroke{Ctrl+C}. 
\end{enumerate}

\noindent You can paste the text into other programs.

\noindent To copy a text or image link (\acronym{URL}) from a page:

\begin{enumerate}
  \item Position the pointer over the link or image.
  \item Right-click on the link or image to open a popup menu.
  \item Choose \menu{Copy Link Location}.% Used to be "\item Choose \menu{Copy Link Location} or \menu{Copy Image Location}." - JasonCook599
\end{enumerate}

\noindent You can paste the link into other programs or into Firefox's Location Bar.

\subsubsection{Saving all or part of a page}

To save an entire page in Firefox:

\begin{enumerate}
  \item Choose \menu{File \then Save Page As} from the menubar. Firefox should
open the \window{Save As} window.
  \item Choose a location for the saved page.
  \item Type a file name for the page, and click \button{Save}.
\end{enumerate}

\noindent To save an image from a page:

\begin{enumerate}
  \item Position the mouse pointer over the image.
  \item Right-click on the image to display a popup menu.
  \item Choose \menu{Save Image As}. Firefox should open the 
\window{Save Image} window.
  \item Choose a location for the saved image.
  \item Enter a file name for the image and click \button{Save}.
\end{enumerate}

\subsection{Changing your homepage}

By default, Firefox will show the \textbf{Ubuntu Start Page} when you start
Firefox. If you prefer to view another page when you start Firefox, you 
will need to change your homepage preference.

\screenshot{03-firefox-preferences.png}{ss:firefox-preferences}{You can change Firefox settings in this window.}\vspace{10 mm}

\marginnote{The homepage can also be set by entering the addresses that should be open in the \textfield{Home Page}, with a pipe \dash | \dash separating pages to be opened in a new tab}

\noindent To change your homepage:

\begin{enumerate}
  \item Navigate to the page that you would like to become your new homepage.
  \item Choose \menu{Edit \then Preferences} from the menubar.
  \item In the ``Startup'' section on the \tab{General} tab, which is shown by
default, click on the \button{Use Current Page} button. If you had more than 
one tab open then all the tabs will be opened when Firefox starts. If you prefer to have one page open, close the other tabs and repeat Steps 2-4.
  \item Click \button{Close}.
\end{enumerate}

\subsection{Download settings}

\marginnote{The Downloads window shows the progress of currently downloading files, and lists files downloaded in the past. It can be used to open or re-download files.}

In \menu{Edit \then Preferences} you can change how Firefox behaves with downloads. You can tell Firefox where to place downloaded files, or to ask where each time. You can also set the behavior of Firefox's Downloads window. The Downloads window can be hidden entirely, or set to hide when downloads finish.

\subsection{Bookmarks}

When browsing the web you may want to come back to certain web pages again
without having to remember the \acronym{URL}.

In Firefox, you can create \emph{bookmarks}, which are saved in the web 
browser and which you can use to navigate back to your picked web pages.

\subsubsection{Bookmarking a page}

After navigating to a web page you can save its location by bookmarking it.

There are two ways to bookmark a page:

\begin{itemize}
  \item From the menubar, choose \menu{Bookmarks} and then 
\menu{Bookmark This Page}. A window will open. Provide a descriptive
name for the bookmark, and click on the \button{Done} button.
  \item Press \keystroke{Ctrl+D}. A pop-up will appear. Provide a descriptive
name for the bookmark, and click on the \button{Done} button.
\end{itemize}

\subsubsection{Navigating to a bookmarked page}

To navigate to a bookmarked page, open the \menu{Bookmarks} menu from 
the menubar, and then choose your bookmark's name. Firefox should open the
bookmark in the current tab.

\advanced{You can also press \keystroke{Ctrl+B} to display bookmarks in a 
sidebar on the left side of the browser window. Press \keystroke{Ctrl+B} again
to hide the sidebar.}

\subsubsection{Deleting a bookmark}

If you would like to delete a bookmark that you have previously made,
open the \menu{Bookmarks} menu from the menubar, and then right-click on
your bookmark's name. Firefox should open a popup menu for your bookmark.
Choose the \menu{Delete} option from the menu. Your bookmark should then
be deleted.

\subsection{History}

Whenever you are browsing the web, Firefox is saving your browsing history.
This allows you to come back to a web page that you have recently visited, 
without needing to remember the page's \acronym{URL}, or even bookmarking it.

To see your most recent history, open the \menu{History} menu from
the menubar. The menu should then display several of the most recent web pages
that you were viewing. Choose one of the pages to return to it.

To see the web pages you have visited recently, press \keystroke{Ctrl+H}. Firefox will 
open a ``sidebar'' on the left side of the browser window, that contains
your browsing history, categorized as ``Today,'' ``Yesterday,'' ``Last 7 days,'' ``This month,'' the past 6 months (listed month by month), and finally ``Older than 6 months.''

Click on one of the date categories in the sidebar to expand it. Then it will reveal the 
pages you visited during that period. Then, once you find the page you need, 
click on its title to return to it.

You can also search for a page by its title. Enter a few letters, or a word,
in the \textbf{Search} field at the top of the history sidebar. The sidebar 
should then display a list of web pages whose titles match your search words.
Click on the title of the page you need to return to it.

If you would like to hide the history sidebar again, press \keystroke{Ctrl+H}
again.

\subsection{Clearing private data}

At times, you may want to delete all private data that Firefox stores about
your browsing history. While this data is stored only on your computer,
you may want to remove it if you share access to your computer.

To delete your private data, open the \menu{Tools} menu from the menubar, and
choose \menu{Clear Recent History}. In the drop down list for the
\dropdown{Time range to clear}, choose how far back you would like Firefox
to delete. 

If you would like more control over what you clear, click on the 
\button{Details} text to display a list of options. 

When done, click on the \button{Clear Now} button.

%FUTURE: \subsection{Troubleshooting connection problems}
%Work offline
%DNS issues
%I think that instead if creating a new "acticle" for troubleshooting and DNS issues, we could put the articles in the networking sections. If the articles are not written they should be written there as this is not the only app that uses the Internet

\subsection{Using a different web browser}

\screenshot{03-preferred-applications.png}{ss:preferred-applications}{You can change the default browser with the "Preferred Applications" utility. To use it, open the \menu{System \then Preferences \then Preferred Applications.}}

If you install a different web browser on your computer, you may want to use it as the 
default browser when you click on links from emails, instant messages, and other places. 

To change your preferred web browser, open the \menu{System} menu from Ubuntu's
main menubar. Then, choose \menu{System} \then{Preferences} \then{Preferred 
Applications}. Ubuntu should then open the \window{Preferred Applications} 
window.

In the ``Web Browser'' section, choose your new preferred web browser, and 
click \button{Close}.
