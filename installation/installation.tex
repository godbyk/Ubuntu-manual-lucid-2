% Chapter 1 - Benjamin Humphrey

\chapter{Installation}
\label{ch:installation}

\section{Getting Ubuntu}
\marginnote{Many companies (such as \Index{Dell} and \Index{System76}) sell computers with Ubuntu preinstalled. If you already have Ubuntu installed on your computer, feel free to skip to \chaplink{ch:the-ubuntu-desktop}.}
Before you can get started with Ubuntu, you will need to obtain a copy of the Ubuntu installation \acronym{CD}. Some options for doing this are outlined below.

\subsection{Downloading Ubuntu}
\index{Ubuntu!downloading|(}
The easiest and most common method for getting Ubuntu is to download the Ubuntu \acronym{CD} image directly from \url{http://www.ubuntu.com}. Head to the website and click the ``Download Ubuntu'' link at the top. Select the nearest download location to you in the drop-down box (to ensure maximum download speed), then click ``Begin Download.''

\subsubsection{32-bit vs 64-bit}
\marginnote{\emph{32-bit} and \emph{64-bit} are types of processor architectures. 64-bit is newer, and most recent computers will come with a 64-bit capable processor. See \chaplink{ch:learning-more} for more information.}
You may notice the words ``Ubuntu Desktop 10.04 (32-bit)'' underneath the default download button on the website. If you are unsure what 32-bit means, don't worry. 32-bit will work on most computers, so if in doubt, simply proceed with the download. However, if you know that your computer is capable of using 64-bit software, you may wish to try the 64-bit version instead. To do this, click on ``Alternative download options'' and make your selection.


\subsubsection{Downloading Ubuntu as a torrent} 

When a new version of Ubuntu is released, sometimes the \glspl{server} can get clogged up with large numbers of people downloading or upgrading at the same time. If you are familiar with using torrents, you may wish to download the torrent file by clicking ``Alternative download options,'' and obtain your copy of the \acronym{CD} image this way instead. You may see significant improvements to your download speed, and will also be helping to spread Ubuntu to other users worldwide. Again, if you are unsure how to use torrents, you can use the default download options on the website. 
\marginnote{\textbf{Torrents} are a way of sharing files and information around the Internet via peer-to-peer file sharing. When a new version of Ubuntu is released, the Ubuntu servers can become very busy. If you know how to use torrents, we recommend that you download the \acronym{CD} image this way to take the load off the servers during periods of high demand.}

\subsection{Burning the \acronym{CD} image}
\marginnote{While the 64-bit version of Ubuntu is referred to as the ``AMD64'' version, it will work on Intel, AMD, and other compatible 64-bit processors.}
Once your download is complete, you will be left with a file called \emph{ubuntu-10.04-desktop-i386.iso} or similar (\emph{i386} here in the filename refers to the 32-bit version. If you had downloaded the 64-bit version, the filename would contain \emph{amd64} instead). This file is a \acronym{CD} image\dash a bit like a snapshot of the contents of a \acronym{CD}\dash which you will need to burn to a \acronym{CD}. To find out how to burn a \acronym{CD} image on your computer, refer to your operating system's or manufacturer's support documentation. You can also find detailed instructions at \url{https://help.ubuntu.com/community/BurningIsoHowto}.

\subsection{Ordering a free \acronym{CD}}

\marginnote{You will be required to create a free online account with Launchpad before you can place your \acronym{CD} order. Once you have Ubuntu installed and running, you will need this account again for use with all \textbf{Ubuntu One} services. See \chaplink{ch:default-applications} for more information on Ubuntu One.}
Alternatively, a free \acronym{CD} can be ordered from Canonical. This option may be preferred if you don't have access to a \acronym{CD} burner, have limited bandwidth, or have a slow Internet connection. There are no shipping costs or other charges when you order an Ubuntu \acronym{CD}. Simply visit \url{http://shipit.ubuntu.com} to request your free Ubuntu Desktop Edition \acronym{CD}.

The \acronym{CD} usually takes two to six weeks to arrive, depending on your location and the current demand. If you would rather start using Ubuntu sooner, you may prefer to follow the instructions above for downloading the \acronym{CD} image, and then burn it to a disc instead.

\marginnote{It is possible to purchase Ubuntu on \acronym{CD} from some computer stores or online shops. Have a look around your local area or on the Internet to see if someone is selling it near you. Even though Ubuntu is free software, it's not illegal for people to sell it.}
\index{Ubuntu!downloading|)}

\subsection{The Live \acronym{CD}}
\label{sec:livecd}

The Ubuntu \acronym{CD} functions not only as an installation \acronym{CD} for putting Ubuntu onto your computer, but also as a Live \acronym{CD}. A Live \acronym{CD} allows you to test Ubuntu without making any permanent changes to your computer by running the entire operating system straight from the \acronym{CD}. 

Your computer reads information from a \acronym{CD} at a much slower speed than it can read information off of a hard drive. Running Ubuntu from the Live \acronym{CD} also occupies a large portion of your computer's memory, which would usually be available for programs to access when Ubuntu is running from your hard drive. The Live \acronym{CD} experience will therefore feel slightly slower than it does when Ubuntu is actually installed on your computer. However, running Ubuntu from the \acronym{CD} is a great way to test things out and allows you to try the default applications, browse the Internet, and get a general feel for the operating system. It's also useful for checking that your computer hardware works properly in Ubuntu and that there are no major compatibility issues.

\marginnote{In some cases, your computer will not recognize that the Ubuntu \acronym{CD} is present as it starts up, and will start your existing operating system instead. Generally, it means that the priority given to \emph{devices} when your computer is starting needs to be changed. For example, your computer might be set to look for information from your hard drive first, and then to look for information on a \acronym{CD} second. In order to run Ubuntu from the Live \acronym{CD}, we want it to look for information from a \acronym{CD} first. Changing your \emph{boot priority} is beyond the scope of this guide. If you need assistance to change the boot priority, see your computer manufacturer's documentation for more information.}
To try out Ubuntu using the Live \acronym{CD}, insert the Ubuntu \acronym{CD} into your \acronym{CD} drive and restart your computer. Most computers are able to detect when a bootable \acronym{CD} is present in your drive at startup\dash that is, a \acronym{CD} that will temporarily take precedence over your usual operating system. As your computer starts, it will run whatever information is stored on this bootable \acronym{CD}, rather than the information stored on your hard drive which your computer usually looks for.

Once your computer finds the Live \acronym{CD}, and after a quick loading screen, you will be presented with the ``Welcome'' screen. Using your mouse, select your language from the list on the left, then click the button labeled \button{Try Ubuntu 10.04}. Ubuntu will then start up, running straight from the Live \acronym{CD}.

%\screenshotTODO{Ubuntu Live \acronym{CD} Welcome screen}
\screenshot{01-live-cd-welcome.png}{ss:live-cd-welcome}{The ``Welcome'' screen allows you to choose your language.}

Once Ubuntu is up and running, you will see the default desktop. We will talk more about how to actually use Ubuntu in \chaplink{ch:the-ubuntu-desktop}, but for now, feel free to test things out. Open some programs, change settings and generally explore\dash any changes you make will not be saved once you exit, so you don't need to worry about accidentally breaking anything. 

When you are finished exploring, restart your computer by clicking the ``Power'' button in the top right corner of your screen (circle with a line through the top) and then select \menu{Restart.} Follow the prompts that appear on screen, including removing the Live \acronym{CD} and pressing \keystroke{Enter} when instructed, and then your computer will restart. As long as the Live \acronym{CD} is no longer in the drive, your computer will return to its original state as though nothing ever happened!

\section{Minimum system requirements}
\index{system requirements|(}
\marginnote{The majority of computers in use today will meet the requirements listed here; however, refer to your computer's documentation or speak to the manufacturer if you would like more information.}
Ubuntu runs well on most computer systems. If you are unsure whether it will work on your computer, the Live \acronym{CD} is a great way to test things out first. For the more technically minded, below is a list of hardware specifications that your computer should meet as a minimum requirement.
\begin{itemize}
  \item 700~MHz x86 processor
  \item 256~\acronym{MB} of system memory (\acronym{RAM})
  \item 3~\acronym{GB} of disk space
  \item Graphics card capable of 1024$\times$768 resolution
  \item Sound card
  \item A network or Internet connection
\end{itemize}
\index{system requirements|)}

\section{Installing Ubuntu}

The process of installing Ubuntu is designed to be quick and easy. However, we do realize that some people may find the idea a little daunting. To help you get started, we have included step-by-step instructions below, along with screenshots so you can see how things will look along the way.

\marginnote{Alternatively, you can also use your mouse to double-click the ``Install Ubuntu 10.04'' icon that is visible on the desktop when using the Live \acronym{CD}. This will start the Ubuntu installer.}
If you have already tested out the Ubuntu Live \acronym{CD}, you should now be familiar with the initial ``Welcome'' screen that appears (refer to \seclink{sec:livecd} section above for more information). Again, select your language on the left-hand side, then click the button labeled \button{Install Ubuntu 10.04.}

At least 3~\acronym{GB} of free space on your hard drive is required in order to install Ubuntu; however, 10~\acronym{GB} or more of free space is recommended. This will ensure that you will have plenty of room to install extra programs later on, as well as store your own documents, music, and photos. 
\marginnote{There are two other options presented on the ``Welcome'' screen: \textbf{release notes} and \textbf{update this installer}. Clicking on the blue underlined \textbf{release notes} will open a web page containing any important information regarding the current version of Ubuntu. Clicking \textbf{update this installer} will search the Internet for any updates to the Ubuntu Live \acronym{CD} that may have been released since your \acronym{CD} was created.}

\subsection{Getting started}
To get started, place the Ubuntu \acronym{CD} in your \acronym{CD} drive and restart your computer booting into Ubuntu. When the welcome screen is displayed select your language and click the \button{Install Ubuntu 10.04}.

The next screen will display a world map. Using your mouse, click your location on the map to tell Ubuntu where you are. Alternatively, you can use the \dropdown{drop-down lists} underneath. This allows Ubuntu to set up your system clock and other location-based features. Click \button{Forward} when you are ready to move on.

\screenshot{01-where-are-you.png}{ss:where-are-you}{Tell Ubuntu your location.}

Next, you need to tell Ubuntu what keyboard you are using. Usually, you will find the suggested option is satisfactory. If you are unsure, you can click the \button{Guess} button to have Ubuntu work out the correct choice by asking you to press a series of keys. You can also choose your own keyboard layout from the list. If you like, type something into the box at the bottom to make sure you are happy with your selection, then click \button{Forward} to continue.

%\screenshotTODO{Installation: Keyboard screen}
\screenshot{01-keyboard-setup.png}{ss:keyboard-setup}{Check your keyboard layout is correct.}

\subsection{Prepare disk space}

This next step is often referred to as \gls{partitioning}. Partitioning is the process of allocating portions of your hard drive for a specific purpose. When you create a \gls{partition}, you are essentially dividing up your hard drive into sections that will be used for different types of information. Partitioning can sometimes seem complex to a new user; however, it does not have to be. In fact, Ubuntu provides you with some options that greatly simplify this process.

%\screenshotTODO{Installation: Partitioning screen}
\screenshot{01-partition.png}{ss:partition}{Choose where you would like to install Ubuntu.}

\subsubsection{Erase and use the entire disk}
\marginnote{Many people installing Ubuntu for the first time currently use another operating system on their computer, such as Windows \acronym{XP}, Windows Vista, Windows 7, or Mac \acronym{OS X}. Ubuntu provides you with the option of either \emph{replacing} your
existing operating system altogether, or installing Ubuntu alongside your
existing system. The latter is called \emph{dual-booting}. Whenever you turn on
or restart your computer, you will be given the option to select which operating system you want to use for that session.}
Use this option if you want to erase your entire disk. This will delete any existing operating systems that are installed on that disk, such as Windows \acronym{XP}, and install Ubuntu in its place. This option is also useful if you have an empty hard drive, as Ubuntu will automatically create the necessary partitions for you.

\subsubsection{Guided partitioning}
If you already have another operating system installed on your hard drive, and want to install Ubuntu alongside it, choose the \radiobutton{Install them side by side, choosing between them each startup} option.

Ubuntu will automatically detect the other operating system and install Ubuntu alongside it. For more complicated \gls{dual-booting} setups, you will need to configure the partitions manually.

\subsubsection{Specifying partitions manually}
\marginnote{Ubuntu installs a \textbf{home folder} where your personal files and configuration data are located by default. If you choose to have your home folder on a separate partition, then in the event that you decide to reinstall Ubuntu or perform a fresh upgrade to the latest release, your personal files and configuration data won't be lost.}
This option is for more advanced users and is used to create special partitions, or format the hard drive with a filesystem different to the default one. It can also be used to create a separate \texttt{/home} partition. This can be very useful in case you decide to reinstall Ubuntu, as it allows you to format and reinstall the operating system, whilst keeping all your personal files and program settings intact in a separate partition. 

Because this is quite an advanced task, we have omitted the details from this edition of \emph{Getting Started with Ubuntu.} You can see more information and detailed instructions on partitioning here: \url{https://help.ubuntu.com/community/HowtoPartition}.

Once you are happy with the way the partitions are going to be set up, click the \button{Forward} button at the bottom to move on.

\subsection{Enter your details}

Ubuntu needs to know some information about you so it can set up the primary login account on your computer. Your name will appear on the login screen as well as the \gls{MeMenu}, which will be discussed further in \chaplink{ch:the-ubuntu-desktop}.

On this screen you will need to tell Ubuntu:

\begin{itemize}
	\item your real name,
	\item your desired username,
	\item your desired password,
	\item what you want to call your computer,
	\item how you want Ubuntu to log you in.
\end{itemize}

%\screenshotTODO{Installation: Who are you? screen}
\screenshot{01-who-are-you.png}{ss:who-are-you}{Setup your user account.}

Type in your full name under ``What is your name?''. The next text field is where you select a username for yourself, and is the name that will be displayed at the Ubuntu login screen when you turn on your computer. You will see this is automatically filled in for you with your first name. Most people find it easiest to stick with this. However, it can be changed if you prefer. 

\marginnote{Although you can choose your preferred username and computer name, you are required to stick with Latin letters, numbers, hyphens, and dots. You will receive a warning if non-acceptable symbols or other characters are entered, and until this is altered you will be unable to progress to the next screen.}
Next, choose a password and enter it into the first password field on the left, then type the same again into the right field to verify. When both passwords match, a strength rating will appear on the right that will tell you whether your password is ``too short,'' ``weak,'' ``fair,'' or ``strong.'' You will be able to continue the installation process regardless of your password strength, but for security reasons it is best to choose a strong one. This is best achieved by having a password that is at least six characters long, and is a mixture of letters, numbers, symbols, and uppercase/lowercase. For extra security, avoid obvious passwords like your birth date, spouse's name, or the name of your pet. 

Now you need to decide on your computer's name. Again, this will be filled in for you automatically using the login name you entered above (it will say something like ``john-desktop'' or ``john-laptop.''). However, it can be changed if you prefer. Your computer name will mainly be used for identifying your computer if you are on a home or office network with multiple computers. To learn more about setting up a network, refer to \chaplink{ch:default-applications}.

Finally, at the bottom of this screen you have three options to choose from regarding how you want to log in to Ubuntu.

\subsubsection{Log in automatically}

Ubuntu will log in to your primary account automatically when you start up the computer so you won't have to enter your username and password. This makes your login experience quicker and more convenient, but if privacy or security are important to you, this option is not recommended. Anyone who can physically access your computer will be able to turn it on and also access your files.  

\subsubsection{Require my password to login}

This option is selected by default, as it will prevent unauthorized people from accessing your computer without knowing the password you created earlier. This is a good option for those who, for example, share their computer with other family members. Once the installation process has been completed, an additional login account can be created for each family member. Each person will then have their own login name and password, account preferences, Internet bookmarks, and personal storage space. 

\subsubsection{Require my password to login and decrypt my home folder}

This option provides you with an extra layer of security. Your home folder is where your personal files are stored. By selecting this option, Ubuntu will automatically enable encryption on your home folder, meaning that files and folders must be \gls{decrypted} using your password before they can be accessed. Therefore if someone had physical access to your hard drive (for example, if your computer was stolen and the hard drive removed), they would still not be able to see your files without knowing your password. 
\warning{If you choose this option, be careful not to enable automatic login at a later date. It will cause complications with your encrypted home folder, and will potentially lock you out of important files.}

\subsection{Confirm your settings and begin installation}

The last screen summarizes your install settings, including any changes that will be made to the partitions on your hard drive. Note the warning about data being destroyed on any removed or formatted partitions\dash if you have important information on your hard drive that is not backed up, now would be a good time to check that you have set up your partitions correctly.
\marginnote{You should not need to click the \button{Advanced} button unless you wish to change your bootloader settings or network proxy. These are more advanced tasks and beyond the scope of this guide.} 
Once you have made sure that all the settings are correct, click on \button{Install} to begin the installation process.

%\screenshotTODO{Installation: Confirmation screen}
\screenshot{01-confirmation.png}{ss:confirmation}{Check that everything is set up right before Ubuntu is installed.}


Ubuntu will now install. As the installation progresses, a slideshow will give you an introduction to some of the default applications included with Ubuntu. These applications are covered in more detail in \chaplink{ch:default-applications}.

%\screenshotTODO{Installation: First slide in the slideshow}
\screenshot{01-first-slide.png}{ss:first-slide}{The first slide in the installation slideshow.}


After approximately twenty minutes, the installation will complete and you will be able to click \button{Restart Now} to restart your computer and start Ubuntu. The \acronym{CD} will be ejected, so remove it from your \acronym{CD} drive and press \keystroke{Enter} to continue.

%\screenshotTODO{Installation: Restart now dialog}
\screenshot{01-restart-now.png}{ss:restart-now}{You are now ready to restart your computer.}

Wait while your computer restarts, and you will then see the login window (unless you selected automatic login).

%\screenshotTODO{Installation: The Ubuntu login window}
\screenshot{01-ubuntu-login.png}{ss:ubuntu-login}{The Ubuntu login window.}

Click your username and enter your password, then press \keystroke{Enter} or click \button{Log in}. You will then be logged in to Ubuntu and will be presented with your new desktop!

