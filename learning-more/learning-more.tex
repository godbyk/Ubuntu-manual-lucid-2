% Chapter 9 - Josh Holland - first half only

\chapter{Learning more}
\label{ch:learning-more}

\section{What else can I do with Ubuntu?}

By now, you should be able to use your Ubuntu desktop for all your daily activities \dash such as browsing the web and editing documents. But you may be interested in learning about other versions of Ubuntu you can integrate into your digital lifestyle. In this chapter, we'll provide you with more detail about versions of Ubuntu that are specialized for certain tasks. But first, we'll first discuss the technologies that make Ubuntu a powerful collection of software.

\section{Open Source Software}

Ubuntu is open source software. Open source software differs from proprietary software \dash software whose source code is patented and is therefore not freely available for modification or distribution by anyone but the rightsholder.\marginnote{The \textbf{source code} of a program is the collection files that have been written in a computer language to make the program.} Microsoft Windows and Adobe Photoshop are examples of proprietary software. \marginnote{\textbf{Proprietary software} is software that cannot be copied, modified, or distributed freely.}

Unlike proprietary software programs, Ubuntu is specifically licensed to promote sharing and collaboration. The legal rules governing Ubuntu's production and distribution ensure that anyone can obtain, run, or share it for any purpose they wish. Computer users can modify open source software to suit their individual needs, share it, improve it, or translate it into other languages \dash provided they release these modifications so others can do the same. In fact, the terms of many open source licensing agreements make it illegal not to do so.

Because open source software is developed by large communities of programmers distributed throughout the globe, it benefits from rapid development cycles and speedy security releases (in the event that someone discovers bugs in the software). In other words, open source software is updated, enhanced, and made more secure every day as programmers all over the world continue to improve it.

Aside from these technical advantages, open source software also has economic benefits. While users must adhere to the terms of an open source licensing agreement when installing and using Ubuntu, for instance, they needn't pay to obtain this license.  While not all open source software is free of monetary costs, much is.

To learn more about open source software, see the Open Source Initiative's open source definition, available at \url{http://www.opensource.org/docs/definition.php}.

\section{Distribution families}

Ubuntu is one of several popular operating systems based on Linux (an open source operating system). While other versions of Linux, or ``distributions,'' may look different from Ubuntu at first glance, they share similar characteristics because of their common roots.\marginnote{A \gls{distribution}, or ``distro,'' is a operating system made from open source programs, bundled together to make them easier to install and use.}

Linux distributions can be divided into two broad families: the Debian family and the Red Hat family.  Each family is named for a distribution on which subsequent distributions are based. For example, ``Debian'' refers to both the name of a distribution as well as the family of distributions derived from Debian. Ubuntu is part of the Debian family of distributions, as are Linux Mint, Xandros, and CrunchBang Linux. Distributions in the Red Hat family include Fedora, openSUSE, and Mandriva.

The most significant difference between Debian-based and Red Hat-based distributions is the system each uses for installing and updating software. These systems are called ``package management systems.''\marginnote{\textbf{Package management systems} are the means by which users can install, remove, and organize software installed on computers with open source operating systems like Ubuntu.} Debian software packages are \acronym{deb} files, while Red Hat software packages are \acronym{rpm} files. For more information about package management, see \chaplink{ch:software-management}.

You will also find distributions that have been specialized for certain tasks. Next, we'll describe these versions of Ubuntu and explain the uses for which each has been developed.

\subsection{Choosing amongst Ubuntu and its derivatives}

Just as Ubuntu is based on Debian, several distributions are subsequently based on Ubuntu. Some of these are made for general use, and each differs with respect to the software included as part of the distribution. Others are designed for specialized uses.

Four derivative distributions are officially recognized and supported by both Canonical and the Ubuntu community.  These are:
\begin{itemize}
    \item \textbf{Ubuntu Netbook Edition}, which is optimized for netbook computers.
    \item \textbf{Kubuntu}, which uses the \acronym{KDE} graphical environment instead of the \acronym{GNOME} environment found in Ubuntu;
    \item \textbf{Edubuntu}, which is designed for use in schools; and
    \item \textbf{Ubuntu Server Edition}, which is designed for use on servers, and typically is not used as a desktop operating system because it doesn't have a graphical interface.
\end{itemize}

Four other derivatives of Ubuntu are available. These include:
\begin{itemize}
    \item \textbf{Xubuntu}, which uses the \acronym{XFCE} graphical environment instead of the \acronym{GNOME} environment found in Ubuntu;
%	\item \textbf{Lubuntu}, which uses the \acronym{LXDE} graphical environment.
    \item \textbf{Ubuntu Studio}, which is designed for creating and editing multimedia; and
    \item \textbf{Mythbuntu}, which is designed for creating a home theater \acronym{PC} with MythTV (an open source digital video recorder).
\end{itemize}

For more information about these derivative distributions, see \url {http://www.ubuntu.com/project/derivatives}.

\subsection{Ubuntu Netbook Edition}
\label{sec:netbook-edition}

Ubuntu Netbook Edition is a version of Ubuntu designed specifically for netbook computers.\marginnote{\textbf{Netbooks} are low-cost, low-power notebook computers designed chiefly for accessing the Internet.} It is optimized for computing devices with small screens and limited resources (like the energy-saving processors and smaller hard disks common among netbooks). Ubuntu Netbook Edition sports a unique interface and features a collection of software applications particularly useful to on-the-go users.

Because many netbooks do not contain \acronym{CD-ROM} drives, Ubuntu Netbook Edition allows users to install it on their computers using \acronym{USB} flash drives. To learn more about using a flash drive to install Ubuntu Netbook Edition on a netbook computer, visit \url{https://help.ubuntu.com/community/Installation/FromImgFiles}.

\subsection{Ubuntu Server Edition}
\label{sec:server-edition}

The Ubuntu Server Edition is an operating system optimized to perform multiuser tasks when installed on servers.\marginnote{A \textbf{server} is a computer that's been configured to manage, or ``serve,'' files many people wish to access.} Such tasks include file sharing and website or email hosting. If you are planning to use a computer to perform tasks like these, you may wish to use this specialized server distribution in conjunction with server hardware.

This manual does not explain the process of running a secure web server or performing other tasks possible with Ubuntu Server Edition. For details on using Ubuntu Server Edition, refer to the manual at \url{http://www.ubuntu.com/server}.

\subsection{Ubuntu Studio}
\label{sec:studio}

This derivative of Ubuntu is designed specifically for people who use computers to create and edit multimedia projects. For instance, it features applications to help users manipulate images, compose music, and edit video. While users can install these applications on computers running the desktop version of Ubuntu, Ubuntu Studio makes them all available immediately upon installation.

If you would like to learn more about Ubuntu Studio (or obtain a copy for yourself), visit \url{http://ubuntustudio.org/home}.

\subsection{Mythbuntu}
\label{sec:mythbuntu}

Mythbuntu allows users to turn their computers into entertainment systems. It helps users organize and view various types of multimedia content such as movies, television shows, and video podcasts. Users with \acronym{TV} tuners in their computers can also use Mythbuntu to record live video and television shows.

To learn more about Mythbuntu, visit \url{http://www.mythbuntu.org/}.


\section{32-bit or 64-bit?}
\label{sec:32-or-64}

As mentioned earlier in this manual, Ubuntu and its derivatives are available in two versions: 32-bit and 64-bit. This difference refers to the way computers process information.  Computers capable of running 64-bit software are able to process more information than computers running 32-bit software; however, 64-bit systems require more memory in order to do this.  Nevertheless, these computers gain performance enhancements by running 64-bit software.

Why choose one over another?  Pay attention to the version you select in the following cases:
\begin{itemize}
    \item If your computer is fairly old (made before 2007), then you may want to install the 32-bit version of Ubuntu. This is also the case for most netbooks.
    \item If your computer has more than 4~\acronym{GB} of memory (\acronym{RAM}), then you may need to install the 64-bit version in order to use all the installed memory.
\end{itemize}

\section{Finding additional help and support}

This guide is not intended to be an all-encompassing resource filled with everything you'll ever need to know about Ubuntu. Because \emph{Getting Started with Ubuntu 10.04} could never answer all your questions, we encourage you to take advantage of Ubuntu's vast community when seeking further information, troubleshooting technical issues, or asking questions about your computer.  Below, we'll discuss a few of these resources \dash located both inside the operating system and on the Internet \dash so you can learn more about Ubuntu or other Linux distributions.

\subsection{System help}

If you need additional help when using Ubuntu or its applications, click the \textbf{Help} icon on the top panel, or navigate to \menu{System \then Help and Support.}  Ubuntu's built-in help guide covers a broad range of topics in great detail.

\subsection{Online Ubuntu help}

The Ubuntu Documentation team has created and maintains a series of wiki pages designed to help both new and experienced users learn more about Ubuntu. You can access these at \url{http://help.ubuntu.com}.

\subsection{The Ubuntu Forums}

The Ubuntu Forums are the official forums of the Ubuntu community.  Millions of Ubuntu users use them daily to seek help and support from one another. You can create an Ubuntu Forums account in minutes. To create an account and learn more about Ubuntu from community members, visit \url{http://ubuntuforums.org}.

\subsection{Launchpad Answers}

Launchpad, an open source code repository and user community, provides a question and answer service that allows anyone to ask questions about any Ubuntu-related topic. Signing up for a Launchpad account requires only a few minutes. Ask a question by visiting Launchpad at \url{https://answers.launchpad.net/ubuntu/+addquestion}.

\subsection{Live chat}

If you are familiar with Internet relay chat (\acronym{IRC}), you can use chat clients such as \application{XChat} or \application{Pidgin} to join the channel \#ubuntu on irc.freenode.net. Here, hundreds of user volunteers can answer your questions or offer you support in real time.

\marginnote{In addition to official Ubuntu and community help, you will often find third-party help available on the Internet. While these documents can often be great resources, some could be misleading or outdated. It's always best to verify information from third-party sources before taking their advice.}

\subsection{LoCo teams}

Within the Ubuntu community are dozens of local user groups called ``LoCo teams.'' Spread throughout the world, these teams offer support and advice, answer questions and promote Ubuntu in their communities by hosting regular events. To locate and contact the LoCo team nearest you, visit \url{http://loco.ubuntu.com/}.

\subsection{Community support}

If you've exhausted all these resources and still can't find answers to your questions, visit Community Support at \url{http://www.ubuntu.com/support/community}.
