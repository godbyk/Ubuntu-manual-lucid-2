% Prologue - Benjamin Humphrey

\chapter{Prologue}
\label{ch:prologue}

\section{Welcome}

Welcome to \emph{Getting Started with Ubuntu}, an introductory guide written to help new users get started with Ubuntu. % could possibly have Team logo here, right aligned

Our goal is to cover the basics of Ubuntu (such as installation and working with the desktop) as well as guide you through some of the most popular applications. We designed this guide to be simple to follow, with step-by-step instructions and plenty of screenshots, allowing you to discover the potential of your new Ubuntu system even if you are a novice computer user or are migrating from another operating system for the first time.

Please bear in mind that this guide is still very much a work in progress and always will be. It is written specifically for Ubuntu 10.04 \acronym{LTS}, and although we have aimed to not limit our instructions to this version, it is unavoidable that some things will change over the life of Ubuntu. Whenever a new version of Ubuntu is released, we will incorporate any changes into our guide, and make a new version available at \url{http://www.ubuntu-manual.org}.

\emph{Getting Started with Ubuntu 10.04} is not intended to be a comprehensive Ubuntu instruction manual. It is more like a quick-start guide that will get you doing the things you need to do with your computer quickly and easily, without getting bogged down with technical details.


If you are after more detail, there are excellent resources available at \url{http://help.ubuntu.com}. Ubuntu's built-in system documentation is also very useful for accessing help on specific topics, and can be found by clicking \menu{System\then Help and Support} in Ubuntu.
\marginnote{More information about Ubuntu's online and system documentation can be found in \chaplink{ch:learning-more}.}%
If something isn't covered here, chances are you will find the information you are looking for in one of those locations. We will try our best to include links to more detailed help wherever we can.

\section{Ubuntu philosophy}
\index{Ubuntu!philosophy of|(}
\index{Ubuntu!definition of}
The term ``Ubuntu'' is a traditional African concept that originated from the Bantu languages of southern Africa. It can be described as a way of connecting with others\dash living in a global community where your actions affect all of humanity. Ubuntu is more than just an operating system: it is a community of people that come together voluntarily to collaborate on an international software project that aims to deliver the best possible user experience.
\index{Ubuntu!philosophy of|)}

\subsection{The Ubuntu promise}
\index{Ubuntu promise}

\begin{itemize}
  \item Ubuntu will always be free of charge, along with its regular
    enterprise releases and security updates.

  \item Ubuntu comes with full commercial support from \gls{Canonical} and
    hundreds of companies from across the world.

  \item Ubuntu provides the best translations and accessibility features
    that the free software community has to offer.

  \item Ubuntu's core applications are all free and open source. We want you
    to use free and open source software, improve it, and pass it on.
\end{itemize}


\section{A brief history of Ubuntu}
\index{Ubuntu!history of|(}

Ubuntu was conceived in 2004 by \Index[Shuttleworth, Mark]{Mark Shuttleworth}, a successful South African entrepreneur, and his company \Index[Canonical]{\gls{Canonical}}. \marginnote{Canonical is the company that provides financial and technical support for Ubuntu. It has employees based around the world who work on developing and improving the operating system, as well as reviewing work submitted by volunteer contributors. To learn more about Canonical, go to \url{http://www.canonical.com}.} Shuttleworth recognized the power of Linux and open source, but was also aware of weaknesses that prevented mainstream use. 

\Index[Shuttleworth, Mark]{Shuttleworth} set out with clear intentions to address these weaknesses and create a system
that was easy to use, completely free (see \chaplink{ch:learning-more} for the complete definition of ``free''), and could compete with other mainstream operating systems. 
With the \Index{Debian} system as a base, \Index[Shuttleworth, Mark]{Shuttleworth} began to build Ubuntu. Using his own funds at first, installation \acronym{CD}s were pressed and shipped worldwide at no cost to the end user. Ubuntu spread quickly, the size of the community rapidly increased, and it soon became the most popular Linux \gls{distribution} available. 

With more people working on the project than ever before, Ubuntu continues to see improvement to its core features and hardware support, and has gained the attention of large organizations worldwide. For example, in 2007, \Index{Dell} began a collaboration with \Index{Canonical} to sell computers with Ubuntu preinstalled. Additionally, in 2005, the French Police began to transition their entire computer infrastructure to a variant of Ubuntu\dash a process which has reportedly saved them ``millions of euros'' in licensing fees for Microsoft Windows. By the year 2012, the French Police anticipates that all of their computers will be running Ubuntu. \Index{Canonical} profits from this arrangement by providing technical support and custom-built software.

\marginnote{For information on Ubuntu Server Edition, and how you can use it in your company, visit \url{http://www.ubuntu.com/server/features}.}
While large organizations often find it useful to pay for support services, \Index[Shuttleworth, Mark]{Shuttleworth} has promised that the Ubuntu desktop system will always be free. As of 2010, Ubuntu is installed on nearly 2\% of the world's computers. This equates to millions of users worldwide, and is growing each year.
\index{Ubuntu!history of|)}

\subsection{What is Linux?}
\index{Linux|(}
Ubuntu is built on the foundation of Linux, which is a member of the \Index{Unix} family. \Index{Unix} is one of the oldest types of operating systems and has provided reliability and security in professional applications for almost half a century. Many servers around the world that store data for popular websites (such as YouTube and Google) run some variant of a \Index{Unix} system. The Linux \Index{kernel} is best described as the core, or almost the brain, of the operating system. 

The Linux \Index{kernel} is the shift manager of the operating system; it is responsible for allocating memory and processor time. It can also be thought of as the program which mangages any and all programs on the computer itself.

\marginnote{While modern graphical \glspl{desktop environment} have generally replaced early command-line interfaces, the command line can still be a quick and efficient way of performing many tasks. See \chaplink{ch:command-line} for more information, and \chaplink{ch:the-ubuntu-desktop} to learn more about \gls{GNOME} and other desktop environments.} 
Linux was designed from the ground up with security and hardware compatibility in mind, and is currently one of the most popular \Index{Unix}-based operating systems. One of the benefits of Linux is that it is incredibly flexible and can be configured to run on almost any device\dash from the smallest micro-computers and cellphones to larger super-computers. 
\Index{Unix} was entirely command line--based until graphical user interfaces (\glspl{GUI}) began to emerge in the early 1990s.

\marginnote{A \emph{desktop environment} is a sophisticated and integrated user interface that provides the basis for humans to interact with a computer using a monitor, keyboard and a mouse.}
These early \acronym{GUI}s were difficult to configure and clunky at best, and generally only used by seasoned computer programmers. In the past decade, however, graphical user interfaces have come a long way in terms of usability, reliability, and appearance. Ubuntu is just one of many different Linux \emph{distributions}, \marginnote{To learn more about Linux distributions, see \chaplink{ch:learning-more}.} and uses one of the more popular graphical desktop environments called \acronym{GNOME}.%
\index{Linux|)}

\section{Is Ubuntu right for you?}

New users to Ubuntu may find that it takes some time to feel comfortable when trying a new operating system. You will no doubt notice many similarities to both Microsoft Windows and Mac \acronym{OS~X}, as well as some differences. Users coming from Mac \acronym{OS~X} are more likely to notice similarities due to the fact that both Mac \acronym{OS~X} and Ubuntu originated from \Index{Unix}. 

\marginnote{A popular forum for Ubuntu discussion and support is the \Index{Ubuntu Forums}, \url{http://ubuntuforums.org}.}
Before you decide whether or not Ubuntu is right for you, we suggest giving yourself some time to grow accustomed to the way things are done in Ubuntu. You should expect to find that some things are different from what you are used to. We also suggest taking the following into account:

\begin{itemize}
    \item \textbf{Ubuntu is community based.} That is, Ubuntu is made, developed, and maintained by the community. Because of this, support is probably not available at your local computer store. Fortunately, the Ubuntu community is here to help. There are many articles, guides, and manuals available, as well as users on various Internet forums and Internet Relay Chat (\acronym{IRC}) rooms that are willing to help out beginners. Additionally, near the end of this guide, we include a troubleshooting chapter: \chaplink{ch:troubleshooting}.

    \item \textbf{Many applications designed for Microsoft Windows or Mac \acronym{OS~X} will not run on Ubuntu.} 
    For the vast majority of everyday computing tasks, there are suitable
    alternative applications available in Ubuntu. However, many professional
    applications (such as the Adobe Creative Suite) are not developed to work
    with Ubuntu. \marginnote{To learn more about \gls{dual-booting}
    (running Ubuntu side-by-side with another operating system), see
    \chaplink{ch:installation}. For more information on Wine, go to
    \url{http://www.winehq.org/}.} If you rely on commercial software that is not compatible with
    Ubuntu, yet still want to give Ubuntu a try, you may want to consider
    \gls{dual-booting}. Alternatively, some applications developed
    for Windows will work in Ubuntu with a program called \application{Wine}.

    \item \textbf{Many commercial games will not run on Ubuntu.} If you are a heavy gamer, then Ubuntu may not be for you. Game developers usually design games for the largest market, which leads to larger profits. Since Ubuntu's market share is not as substantial as Microsoft's Windows or Apple's Mac \acronym{OS~X},
    most game developers will not allocate resources towards making their games compatible with Ubuntu.
    If you just like to play a game every now and then, there is active game development within the community, and many high quality games can be easily installed through \application{Ubuntu Software Center}. \marginnote{See \chaplink{ch:software-management} to learn more about \application{Ubuntu Software Center}.} Additionally, some games developed for Windows will also work in Ubuntu with \application{Wine}.

\end{itemize}

\section{Contact details}

Many people have contributed their time to this project. If you notice any errors or think we have left something out, feel free to contact us. We do everything we can to make sure that this manual is up to date, informative, and professional. Our contact details are as follows:

\bigskip

\textbf{The Ubuntu Manual Team}

\smallskip
Website:
\url{http://www.ubuntu-manual.org/}

\smallskip
Email:
\url{ubuntu-manual@lists.launchpad.net}

\smallskip
\acronym{IRC}: 
\#ubuntu-manual on \url{irc.freenode.net}

\smallskip
Bug Reports:
\url{http://bugs.ubuntu-manual.org}

\section{Conventions used in this book}

The following typographic conventions are used in this book:

\begin{itemize}

\item Button names, menu items, and other \acronym{GUI} elements are set in \textbf{boldfaced type}.

\item Menu sequences are sometimes typeset as \menu{System\then Preferences\then Appearance}, which means, ``Choose the \menu{System} menu, then choose the \menu{Preferences} submenu, and then select the \menu{Appearance} menu item.''

\item \texttt{Monospaced type} is used for text that you type into the computer, text that the computer outputs (as in a terminal), and keyboard shortcuts.

\end{itemize}

