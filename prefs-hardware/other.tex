%Chapter 4: Hardware and Preferences. Section: Other devices
% Written by Sayantan Das - sayantan13@gmail.com


%SCREENSHOTS- most have been removed to save space in the manual we are not putting in all screenshots
%luke Jennings (ubuntujenkins irc) ubuntujenkins@gogolemail.com 


\section{Other devices}
\label{sec:other-devices}
 
\subsection{Firewire}

Firewire is a special type of port that makes use of Firewire technology to 
transfer data. This port is generally used by camcorders and digital cameras. 

If you want to import video from your camcorder you can do so by connecting your camcorder to the Firewire port. 
You will need to install a program called \application{Kino} which is available in the \application{Ubuntu Software Center}. 
\marginnote[-1\baselineskip]{To find out more about Kino, visit \url{http://www.kinodv.org/}.}

\subsection{Bluetooth}

Bluetooth is widely used on \acronym{GPS} devices, mouses, mobile phones, headsets, music players, 
desktops and laptops for data transfer, listening to music, playing games and for 
various other activities. All modern operating systems support Bluetooth and Ubuntu 
is no exception. 

You can access the Bluetooth preferences by left-clicking on the Bluetooth
 icon on the right hand side of the top panel. It is usually located next to the volume icon.
 Left-clicking on the Bluetooth icon opens a popup menu with several choices, such as an option to \menu{Turn off Bluetooth}.

\screenshot{04-bluetooth-left-click.png}{ss:bluetooth-left-click}{The Bluetooth applet menu.} 

The Bluetooth preferences can also be accessed from \menu{System\then Preferences\then Bluetooth}. If you want to setup a new 
device such as a mobile phone to synchronize with your computer, choose the option 
that reads \textbf{Setup new device..}.

Ubuntu will then open a window for new device setup. 
When you click \button{Forward}, Ubuntu will open the second screen which will show you how many Bluetooth devices are present 
within the range of your system. The list of available devices might take a minute 
or so to appear on the screen as your system will be scanning for the devices. 
The scan and display is in real time, which means that every device will be displayed as soon as it is found. Click on the required Bluetooth device from the list of devices. 
Then, select the \acronym{PIN} number by selecting \button{PIN options}. 

Three predefined \acronym{PIN} numbers are available but you can create a custom \acronym{PIN} if you like. 
You will need to enter this \acronym{PIN} on the device you will be pairing with Ubuntu.

Once the device has been paired, Ubuntu will open the ``Setup completed'' screen.

In Ubuntu, your computer is hidden by default for security reasons. 
This means that your Ubuntu system can search other Bluetooth enabled systems but 
they cannot search for your Ubuntu system. You will have to enable the option, if you 
want your Bluetooth device to find your Ubuntu system. You can do this by selecting 
the option ``Make computer discoverable'' in Bluetooth preferences. You can also 
add a fancy name for your Bluetooth-enabled Ubuntu system by changing the text under \textbf{Friendly Name}.
