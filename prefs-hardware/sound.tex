% Chapter 4 : Preferences and Hardware. Section : Sound
% Written by Sayantan Das - sayantan13@gmail.com


%SCREENSHOTS have been removed to save space in the manual we are not putting in all screenshots
%luke Jennings (ubuntujenkins irc) ubuntujenkins@gogolemail.com 

\label{sec:Preferecs and Hardware:Sound}

\section{Sound}

Ubuntu usually detects the audio hardware of the system automatically during
installation. The audio in Ubuntu is provided by a sound server named
PulseAudio. The audio preferences are easily configurable with the help of a
very easy to use \gls{GUI} which comes preinstalled with Ubuntu.

A volume icon, sitting on the top right corner of the screen, provides quick
access to different audio related functions. Left clicking on the volume icon
shows up a slider button which you can move horizontally to increase/decrease 
volume. Left clicking on the volume icon also allows you to choose between muting 
the sound and Sound Preferences. Selecting \emph{Sound Preferences} opens up another 
window which provides access to sound themes, hardware, input and output preferences. 
Sound Preferences can also be found if you go to \menu{System \then Preferences \then Sound.}

The first tab which shows up by default is \emph{sound themes.} You can disable the 
existing sound theme or configure it with the options available. 

\marginnote{You can add new sound themes by installing them from Software Center (\eg, Ubuntu Studio Sound theme.) 
You will get the installed sound themes from the drop down menu. You can also enable window and button sounds.}
The \emph{hardware tab} will have a list of all the sound cards available in your system. 
Usually there is only one listed, however, if you have a graphics card which supports
\acronym{HDMI} audio it will also show up in the list. This section should be configured only if you are an advanced user.

\marginnote{A microphone is used for making audio/video calls which are supported by applications
like Skype or Empathy. It can also be used for sound recording.}
The third tab is for configuring \emph{input audio.} You will be able to use this section
when you have an inbuilt microphone in your system or if you add an external microphone.

\marginnote{You should note that by default in any Ubuntu installation, the input sound is muted. You will have to manually
unmute to enable your microphone to record sound or use it during audio/video calls.}
You can increase/decrease and mute/unmute input volume from this tab. If there is more 
than one input device, you will see them listed in the white box
which reads \emph{Choose a device for sound input.} 

\marginnote{By default, the volume in Ubuntu is set to maximum during installation.}
The \emph{output tab} is used for configuring the output audio. You can increase/decrease and 
mute/unmute output volume and select your preferred output device. 

\marginnote{If you change your sound output device, it will remain as default.}
If you have more than one output device, it will be listed in the section which reads ``Choose a 
device for sound output.'' The default output hardware, which is automatically detected
by Ubuntu during installation will be selected. 

The \emph{Applications tab} is for changing the volume for running applications. 
This comes in very handy if you have multiple audio programs running, for example, if 
you have Rhythmbox, Totem Movie Player and a web-based video playing at the same
time. In this situation, you will be able to increase/decrease, mute/unmute volume for each application from this tab.

