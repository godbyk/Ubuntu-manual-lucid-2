%Written by Rudolph du Plooy: <rudidu@gmail.com>
%Editied by JasonCook599s
\section{Scanning text and images}

Most of the time, Ubuntu will simply detect your scanner and you should just be able to use it. To scan a document, follow these steps:

\begin{enumerate}
  \item Place what you want to scan on the scanner.
  \item Go to \menu{Applications\then Graphics\then Simple Scan}.
  \item Click \button{Scan}.
  \item Click the \button{Paper Icon} to add a another page.
  \item Click \button{Save} to save.
\end{enumerate}

\subsection{Does my scanner work with Ubuntu?}
There are three ways to see if you scanner works in Ubuntu:

\begin{enumerate}
  \item Simply plug it in.  If it is a newer \acronym{USB} scanner, it is likely that it will just work.
  \item Check \url{https://wiki.ubuntu.com/HardwareSupportComponentsScanners} to find out which scanners work with Ubuntu.
  \item \acronym{SANE} project listing of supported scanners. The \acronym{SANE} (Scanner Access Now Easy) project provides most of the back-ends to the scanning software on Ubuntu.
\end{enumerate}

\subsection{Ubuntu can't find my scanner}
There are a few reason why Ubuntu may give you a ``No devices available message'':
\begin{itemize}
  \item Your scanner is not supported in Ubuntu. The most common type of scanner not supported is old parallel port or Lexmark All-in-One printer/scanner/faxes.
  \item The driver for your scanner is not being automatically loaded.
\end{itemize}

%source: https://help.ubuntu.com/community/ScanningHowTo
