% All of the glossary entries must be in this file.

% Run ``texdoc glossaries'' for information on the \newglossaryentry command.


% Try to keep these entries alphabetized to avoid duplication.

\newglossaryentry{applet}{name={applet}, description={%
  An applet is a small program that runs in a \gls{panel}. Applets provide 
  useful functions such as starting a program, viewing the time, or accessing
  the main menu.}}

\newglossaryentry{Access Point}{name={Access Point}, description={%
An Access Point is a devices that allows for a wireless connection to a wireless network using Wi-Fi, Bluetooth, etc.}}

\newglossaryentry{Canonical}{name={Canonical}, description={%
  Canonical, the financial backer of Ubuntu, provides support for the core
  Ubuntu system. It has over 310 paid staff members worldwide who ensure that
  the foundation of the operating system is stable, as well as checking all the
  work submitted by volunteer contributors.  To learn more about Canonical, go
  to \url{http://www.canonical.com}.}}

\newglossaryentry{cursor}{name={cursor}, description={%
  The blinking cursor that appears after the \gls{prompt} in the \gls{terminal} is used
  to show you where text will appear when you start typing. You can move it around with
  arrow keys on your keyboard.}}

\newglossaryentry{decrypted}{name={decrypted}, description={%
  When you decrypt an encrypted file it becomes \gls{decrypted}, and viewable. Encrypted files on Ubuntu are not
  recognizable in any language, they are just a string of random numbers and letters until
  they are \gls{decrypted} using a password.}}

\newglossaryentry{desktop environment}{name={desktop environment}, description={%
  A generic term to describe a \acronym{GUI} interface for humans to interact with computers. There are many
  desktop environments such as \acronym{GNOME}, \acronym{KDE}, \acronym{XFCE} and \acronym{LXDE} just to name a few.}}

\newglossaryentry{DHCP}{name={\acronym{DHCP}}, description={%
  \acronym{DHCP} stands for \emph{Dynamic Host Configuration Protocol}, it is used by
  a \acronym{DHCP} \gls{server} to assign computers on a network an \acronym{IP} address automatically.}}

\newglossaryentry{dialup connection}{name={dialup connection}, description={%
  A dialup connection is when your computer uses a modem to connect to an \gls{ISP} through
  your telephone line.}}

\newglossaryentry{distribution}{name={distribution}, description={%
  A \gls{distribution} is a collection of software that is already compiled and configured
  ready to be installed. Ubuntu is an example of a distribution.}}

\newglossaryentry{dual-booting}{name={dual-booting}, description={%
  \gls{dual-booting} is the process of being able to choose one of two different operating systems
  currently installed on a computer from the boot menu, once selected your computer will then boot into
  whichever operating system you chose at the boot menu. Dual booting is a generic term, and can refer to more than two operating systems.}}

\newglossaryentry{Ethernet port}{name={Ethernet port}, description={%
  An Ethernet port is what an Ethernet cable is plugged into when you are using
   a \gls{wired connection}.}}

\newglossaryentry{GNOME}{name={\acronym{GNOME}}, description={%
  \acronym{GNOME} (which once stood for \acronym{GNU} Network Object Model Environment) is the
  default desktop environment used in Ubuntu.}}

\newglossaryentry{GUI}{name={\acronym{GUI}}, description={%
  The \acronym{GUI} (which stands for Graphical User Interface) is a type of user interface that allows
  humans to interact with the computer using graphics and images rather than just text.}}

\newglossaryentry{ISP}{name={\acronym{ISP}}, description={%
  \acronym{ISP} stands for \emph{Internet Service Provider}, an \acronym{ISP} is a company that provides
  you with your Internet connection.}}

\newglossaryentry{kernel}{name={kernel}, description={%
  A kernel is the central portion of a Unix-based operating system, responsible
  for running applications, processes, and providing security for the core
  components.}}

\newglossaryentry{Live CD}{name={Live CD}, description={%
  A \gls{Live CD} allows you to try out an operating system before you actually install it, this is useful
  for testing your hardware, diagnosing problems and recovering your system.}}

\newglossaryentry{maximize}{name={\emph{maximize}}, description={%
  When you maximize an application in Ubuntu it will fill the whole desktop, excluding the panels.}}

\newglossaryentry{MeMenu}{name={MeMenu}, description={%
  The MeMenu in Ubuntu 10.04 allows you to manage your presence on social networking services.
  It also allows you to publish status messages to all of your accounts by entering updates into a
  text field.}}

\newglossaryentry{minimize}{name={minimize}, description={%
  When you minimize an open application, the window will no longer be shown. If you click on a minimized
  application's panel button, it will then be restored to its normal state and allow you to interact with it.}}

\newglossaryentry{notification area}{name={notification area}, description={%
  The notification area is an applet on the panel that provides you with all sorts of information
  such as volume control, the current song playing in Rhythmbox, your Internet connection status and email status.}}

\newglossaryentry{output}{name={output}, description={%
  The output of a command is any text it displays on the next line after typing a command and pressing
  enter, \eg, if you type \commandlineapp{pwd} into a terminal and press \keystroke{Enter}, the directory name it displays on the
  next line is the output.}}

\newglossaryentry{package}{name={package}, description={%
  Packages contain software in a ready-to-install format. Most of the time you can use the \gls{Software Center} instead of manually installing packages. Packages have a .deb extension in Ubuntu.}}

\newglossaryentry{panel}{name={panel}, description={%
  A panel is a bar that sits on the edge of your screen. It contains
  \glspl{applet} which provide useful functions such as running programs,
  viewing the time, or accessing the main menu.}}

\newglossaryentry{parameter}{name={parameter}, description={%
  Parameters are special options that you can use with other commands in the terminal to make that
  command behave differently, this can make a lot of commands far more useful.}}

\newglossaryentry{partition}{name={partition}, description={%
  A partition is an area of allocated space on a hard drive where you can put data.}}

\newglossaryentry{partitioning}{name={partitioning}, description={%
  \gls{partitioning} is the process of creating a \gls{partition}.}}

\newglossaryentry{prompt}{name={prompt}, description={%
  The prompt displays some useful information about your computer, it can be customized
  to display in different colors as well as being able to display the time, date and current directory
  as well as almost anything else you like.}}

\newglossaryentry{proprietary}{name={proprietary}, description={%
  Software made by companies that don't release their source code under an open source license.}}

\newglossaryentry{router}{name={router}, description={%
  A router is a specially designed computer that using its software and hardware, routes information from the
  Internet to a network. It is also sometimes called a gateway.}}

\newglossaryentry{server}{name={server}, description={%
  A server is a computer that runs a specialized operating system and provides
  services to computers that connect to it and make a request.}}

\newglossaryentry{shell}{name={shell}, description={%
  The \gls{terminal} gives access to the shell, when you type a command into the terminal and press enter
  the shell takes that command and performs the relevant action.}}

\newglossaryentry{Software Center}{name={Software Center}, description={%
  The Software Center is where you can easily manage software installation and
  removal as well as the ability to manage software installed via Personal Package Archives.}}

\newglossaryentry{Synaptic Package Manager}{name={Synaptic Package Manager}, description={%
  Synaptic Package Manager is a tool that, instead of listing applications (like the Software Center)
  lists individual packages that can then be installed, removed and fixed.}}

\newglossaryentry{terminal}{name={terminal}, description={%
  The terminal is Ubuntu's text only interface, it is a method of controlling some aspects of the operating
  system using only commands entered via the keyboard.}}

\newglossaryentry{wired connection}{name={wired connection}, description={%
  A wired connection is when your computer is physically connected to a \gls{router} or \gls{Ethernet port} with
  a cable, this is the most common connection for desktop computers.}}

\newglossaryentry{wireless connection}{name={wireless connection}, description={%
  A wireless connection involves no cables of any sort and instead uses a wireless signal to communicate with
  either a \gls{router}, access point or computer.}}

\newglossaryentry{Wubi}{name={Wubi}, description={%
  Wubi stands for Windows Ubuntu Installer, and it allows you to install Ubuntu
  inside Windows. See page~\pageref{sec:installation:using-wubi} for more
  information.}}

